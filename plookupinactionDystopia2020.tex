\documentclass[shadesubsections,compress,14pt,mathserif]{beamer}
\usepackage[danish]{babel}	
\usepackage{tikz}
\usetikzlibrary{shapes, positioning}
\usenavigationsymbolstemplate{}
\usepackage{pgfplots}
\usepackage[absolute,overlay]{textpos}

%\usepackage[T1]{fontenc}
%\usepackage{fourier}
% Dokumentets sprog
%\usepackage{mathtools}
%\usepackage{pxfonts}
\usepackage{amssymb, eulervm}
% Class options include: notes, notesonly, handout, trans,
%                        hidesubsections, shadesubsections,
%                        inrow, blue, red, grey, brown

% Theme for beamer presentation.
%\usepackage{beamertheme} 
% Other themes include: beamerthemebars, beamerthemelined, 
%                       beamerthemetree, beamerthemetreebars  
\newcommand{\adv}{\ensuremath{\mathcal A}}
\newcommand{\xor}{\ensuremath{\oplus}}
\newcommand{\plonk}{\ensuremath{\mathcal{P} \mathfrak{lon}\mathcal{K}}}
\newcommand{\F}{\ensuremath{\mathbb F}}
\newcommand{\set}[1]{\ensuremath{\left\{#1\right\}}}
\newcommand{\sett}[2]{\ensuremath{\left\{#1\right\}_{#2}}}
\newcommand{\enc}[1]{\ensuremath{\left[#1\right ]}}
\newcommand{\cm}{\ensuremath{\mathsf{cm}}}
\newcommand{\open}[1]{\ensuremath{\mathsf{open}(#1)}}
\newcommand{\verify}[1]{\ensuremath{\mathsf{verify}(#1)}}
\newcommand{\defeq}{\ensuremath{:=}}
\newcommand{\helper}{\ensuremath{\mathcal{H}}}
\newcommand{\ver}{\ensuremath{\mathcal{V}}}
\newcommand{\prv}{\ensuremath{\mathcal{P}}}
 \newcommand{\polysofdeg}[1]{\F_{< #1}[X]}
 \newcommand{\polys}{\F[X]}
\newcommand{\acc}{{\mathbf{acc}}}
\newcommand{\ideal}{\mathbf{I}}
\newcommand{\gen}{\alpha}

%\setbeamersize{text margin left=3mm,text margin right=3mm} 
\title{\large{Plookup in action}}    % Enter your title between curly braces
\author{\small{Ariel Gabizon  \; Zachary J. Williamson}\\                 % Enter your name between curly braces
                                      } }      % Enter your institute name between curly braces
\date{}                    % Enter the date or \today between curly braces
%\usefonttheme{professionalfonts}
%\usefonttheme[onlymath]{serif}
\begin{document}
\boldmath
% Creates title page of slide show using above information
\begin{frame}
  \titlepage
  \includegraphics{azteclogo.png}
\end{frame}

\begin{frame}
\frametitle{Turbo-PLONK programs {\small (based on PLONK[GWC])} }   % Insert frame title between curly braces
\begin{table}[!htbp]
\[	\begin{tabular}{l|l|l|l|}
	 a_{1} & b_{1}&c_{1}&d_{1}\\ \hline
	  \vdots & \vdots&\vdots & \vdots \\ \hline
	 a_{i} & b_{i}&c_{i}&d_{i}\\ \hline
	 a_{i+1} & b_{i+1}&c_{i+1}&d_{i+1}\\ \hline
% 	  w_{i+1,1} & w_{i+1,2}&w_{i+1,3}& w_{i+1,4}\\ \hline
	  \vdots & \vdots&\vdots & \vdots \\ 

	\end{tabular}
\]
\end{table}

\begin{itemize}
 \item Local low-degree constraints between rows (e.g. $a_{i+1}=b_i^2 + c_i$)
 \item Global equality constraints between any two cells (e.g. $a_{100} = d_2$).
\end{itemize}

\end{frame}
\begin{frame}
\frametitle{ \textbf{Ultra}-PLONK  programs }   % Insert frame title between curly braces
% \begin{table}[!htbp]
% \[	\begin{tabular}{l|l|l|l|}
% 	 a_{1} & b_{1}&c_{1}&d_{1}\\ \hline
% 	  \vdots & \vdots&\vdots & \vdots \\ \hline
% 	 a_{i} & b_{i}&c_{i}&d_{i}\\ \hline
% 	 a_{i+1} & b_{i+1}&c_{i+1}&d_{i+1}\\ \hline
% % 	  w_{i+1,1} & w_{i+1,2}&w_{i+1,3}& w_{i+1,4}\\ \hline
% 	  \vdots & \vdots&\vdots & \vdots \\ 
% 
% 	\end{tabular}
% \]
% \end{table}

\begin{itemize}
 \item Local low-degree constraints between rows (e.g. $a_{i+1}=b_i^2 + c_i$).
 \item Global equality constraints between any two cells (e.g. $a_{100} = d_2$).
 \item \textbf{Lookup constraints} - e.g. $(a_5,b_5,c_5)$ is contained in the rows of a predefined table $T$.
\end{itemize}

\end{frame}
\begin{frame}
\frametitle{Lookup constraints in SNARKs}   % Insert frame title between curly braces
%Since the beginning of time (LFKN, 1989) humanity has been trying to verify prover polynomial evaluations.\\
First used in Arya{\footnotesize [Bootle, Cerulli, Groth, Jakobsen, Maller]}\\ \pause
    \vspace{0.4in}
Plookup [GW20] gives improved efficiency:\\

~$2(|T|+|w|)$ prover group exp\\
    \vspace{0.4in}
$|T|$ - number of rows in table\\
$|w|$ - length of witness
\end{frame}
\begin{frame}
\frametitle{Example: bitwise XOR with ``direct'' table}   % Insert frame title between curly braces
%Since the beginning of time (LFKN, 1989) humanity has been trying to verify prover polynomial evaluations.\\
For row values $(a,b,c)$ want to show $c=a\oplus b$ as $11$-bit strings.\pause

    \vspace{0.2in}
 Use table $T$ of all triplets $(a,b,c)$
  
  s.t. $c=a\;\;\oplus\;\;b$.\pause
  
    \vspace{0.2in}
$|T| = 2^{22}$

\end{frame}
\begin{frame}
\frametitle{Another approach - Sparse representations }   
Table $T_1$ of pairs $(a,a_s)$ - $a$ is 10-bit string, $a_s$ is ``$a$ with zeroes in between bits'' - 
% \begin{block}{Formula}
\begin{equation}
a=  \Sigma a_i \cdot 2^i, a_s =  \Sigma a_i \cdot 4^i
\end{equation}\pause
% \end{block}

Field addition on sparse form now gives bitwise XOR :

$a=(1 1) $
$\;\;\;\;\;b=(1 0) $\\ \pause

$a_s = (0 1 0 1)$\\
$b_s = (0 1 0 0)$\\ \pause
\vspace{0.2in}
$a_s + b_s = (1 0 0 1)$\\ \pause
Odd bits are XORs
\end{frame}
\begin{frame}
\frametitle{Another approach - Sparse representations}   % Insert frame title between curly braces
%Since the beginning of time (LFKN, 1989) humanity has been trying to verify prover polynomial evaluations.\\
After adding in sparse form, can use another lookup to ``decode'' XOR result
$T_2 = \set{c_s,c_{XOR}}$ so
\[c_s=\Sigma  c_i 4^i, c_{XOR}=\Sigma \phi(c_i) 4^i,\]
\[\phi(0)=0,\phi(1)=1,\phi(2)=0,\phi(3)=1\]\pause

\vspace{0.4in}
{\small \textit{Can get AND at same time (see Arya paper)}}
\end{frame}
\begin{frame}
\frametitle{SHA-256 with Sparse representations on Steroids }   
% \textit{Motivation: Suppose all SNARK-friendly hash functions got covid from the SNARK}
\end{frame}
\begin{frame}
$MAJ'$ is one of the two main ``chunks'' of a SHA round:
\begin{itemize}
 \item $a,b,c$ 32-bit values 
\item $>>>$ is right rotation.\pause
\end{itemize}
\vspace{0.2in}
\[MAJ'(a,b,c) :=\]
\[(a >>> 2) \oplus (a >>>13) \oplus (a>>> 22) \oplus MAJ(a,b,c)\]\pause
% Assume we've managed to compute the rotations (addressed next)

\end{frame}
\begin{frame}

We map $a,b,c$ into 16-sparse form:\\
$\Sigma a_i 2^i\rightarrow\Sigma a_i 16^i$ \\ \pause
\vspace{0.2in}
In sparse form we simply add in field:\\
\[4*(  (a>>>2)+(a>>>13)+(a>>>22) ) + (a+b+c)\]\pause
Addition result is ``injective enough'' to retrieve output of $MAJ'$.
\end{frame}
% \begin{frame}
% \frametitle{SHA-256 with sparse representations on Steroids }   % Insert frame title between curly braces
% Since the beginning of time (LFKN, 1989) humanity has been trying to verify prover polynomial evaluations.\\
% One of the two main ``chunks'' of a SHA round:
% \[MAJ'(a,b,c) \defeq  a >>> 2 \oplus a >>>13 \oplus a>>> 22 \oplus MAJ(a,b,c)\]
% $a,b,c$ 32-bit values, $>>>$ is right rotation.
% Assume we've managed to compute the rotations (addressed next)
% 
% We map $a,b,c$ into 16-sparse form:
% \[\sum a_i 2^i\rightarrow\sum a_i 16^i\]
% In sparse form we simply add in field:
% \[4*(  (a>>>2)+(a>>>13)+(a>>>22) ) + (a+b+c)\]
% Addition result is ``injective enough'' to retrieve $MAJ'(a,b,c)$.
% \end{frame}
\begin{frame}
\frametitle{Getting the rotations}   
Split 32-bit $a$ to limbs $(a_2,a_1,a_0)$ of $10,11,11$ bits respectively.\\ \pause

\vspace{0.2in}
We have in total 9 ``rotate contributions'': 3 right-rotates  - $2,13,22$ of the three limbs.\\ \pause

\vspace{0.2in}
\emph{But} only two ``non-trivial'' contributions: $(a_1,13),(a_0,2)$ \\ \pause
both can be computed with a table of right rotate by $2$.\\ \pause

\vspace{0.2in}
In total for $MAJ'$- 3 tables of size  $\leq 2^{11}$

\end{frame}
\end{document}
