\documentclass[shadesubsections,compress,14pt,mathserif]{beamer}
\usepackage[danish]{babel}	
\usepackage{tikz}
\usetikzlibrary{shapes, positioning}
\usenavigationsymbolstemplate{}
\usepackage{pgfplots}
\usepackage[absolute,overlay]{textpos}
\usepackage{amsthm}
%\usepackage[T1]{fontenc}
% \usepackage{fullpage}
% Dokumentets sprog
%\usepackage{mathtools}
%\usepackage{pxfonts}
\usepackage{eulervm}

% Class options include: notes, notesonly, handout, trans,
%                        hidesubsections, shadesubsections,
%                        inrow, blue, red, grey, brown

% Theme for beamer presentation.
%\usepackage{beamertheme} 
% Other themes include: beamerthemebars, beamerthemelined, 
%                       beamerthemetree, beamerthemetreebars  
\newcommand{\adv}{\ensuremath{\mathcal A}}
\newcommand{\F}{\ensuremath{\mathbb F}}
\newcommand{\set}[1]{\ensuremath{\left\{#1\right\}}}
\newcommand{\sett}[2]{\ensuremath{\left\{#1\right\}_{#2}}}
\newcommand{\enc}[1]{\ensuremath{\left[#1\right ]}}
\newcommand{\cm}{\ensuremath{\mathsf{cm}}}
\newcommand{\open}[1]{\ensuremath{\mathsf{open}(#1)}}
\newcommand{\verify}[1]{\ensuremath{\mathsf{verify}(#1)}}
\newcommand{\defeq}{\ensuremath{:=}}
\newcommand{\helper}{\ensuremath{\mathcal{H}}}
\newcommand{\ver}{\ensuremath{\mathcal{V}}}
\newcommand{\prv}{\ensuremath{\mathcal{P}}}
 \newcommand{\polysofdeg}[1]{\F_{< #1}[X]}
 \newcommand{\polys}{\F[X]}
\newcommand{\acc}{{\mathbf{acc}}}
\newcommand{\ideal}{\mathbf{I}}
\newcommand{\gen}{\alpha}

\title{\LARGE{zk-proofs - from novice to master}}    % Enter your title between curly braces
\author{\Large{Ariel Gabizon}}                 % Enter your name between curly braces
\institute{\normalsize{Aztec}}      % Enter your institute name between curly braces
\date{}                    % Enter the date or \today between curly braces
%\usefonttheme{professionalfonts}
%\usefonttheme[onlymath]{serif}
\begin{document}
\boldmath
% Creates title page of slide show using above information
\begin{frame}
  \titlepage
\end{frame}
\note{Talk for 30 minutes} % Add notes to yourself that will be displayed when
                           % typeset with the notes or notesonly class options

%\section[Outline]{}

% Creates table of scontents slide incorporating
% all \section and \subsection commands
\begin{frame}
\emph{``If encryption is  a light switch - on or off, zero-knowledge proofs are a dimmer allowing you to control exactly how much you information expose''} 
\end{frame}

\begin{frame}
  \frametitle{The deck of cards:}   % Insert frame title between curly braces
\textbf{A full deck with red and black cards, face down.}\\
 \vspace{0.4in}
\textbf{I take out a red three of hearts.}
 \vspace{0.4in}
\textbf{How to convince you I took a red card, without showing which one}
\end{frame}
\begin{frame}
  \frametitle{Proving color to the color blind:}   % Insert frame title between curly braces
\textbf{A red and green ball, otherwise indentical}\\
 \vspace{0.4in}
\textbf{How to convince a color-blind friend they are different?.}
 \vspace{0.4in}

\end{frame}

\begin{frame}
  \frametitle{Counting leaves in a tree:}   % Insert frame title between curly braces
How to prove you can instantly count the number of leaves on a tree, without disclosing the number of leaves?
\end{frame}


\begin{frame}
  \frametitle{Visual example: Where's Waldo?}   % Insert frame title between curly braces
 \begin{figure}
  \includegraphics[width=40pt]{waldopic.png}
  \includegraphics[width=250pt]{waldo.jpg}
\end{figure}

\end{frame}


% \begin{frame}
%   \frametitle{Video: the cave}   % Insert frame title between curly braces
% \end{frame}


\begin{frame}
 \frametitle{3-coloring}
 How can we prove to someone we can color a graph with 3 colors without leaking the coloring?
 %Example to use: https://web.mit.edu/~ezyang/Public/graph/svg.html
\end{frame}



% \begin{frame}
%   \frametitle{From interactive to non-interactive}
% \textbf{Fiat-Shamir hueristic:} simulate challenges of the verifier by hash of messages so far
%  \vspace{0.4in}
%   
%   
% \end{frame}
% \begin{frame}
%   \frametitle{From interactive to non-interactive}
% \textbf{Fiat-Shamir hueristic:} simulate challenges of the verifier by hash of messages so far\\
%  \vspace{0.4in}
% \textbf{Homomorphic encryption:} Give challenge in advance in homomorphically encoded form (Craig Gentry video)
%  \vspace{0.4in}
% \end{frame}
% 

\begin{frame}
  \frametitle{ZK + bitcoin: Zero-Knowledge contingent payments \small{(by Greg Maxwell)}}
\textbf{Chicken and egg problem:} Alice has sudoku puzzle solution, Bob wants to buy it - who goes first?.\\
 \vspace{0.4in}
\textbf{ZKCP:} Protocol where money and solution change hands at exactly same time.
 \vspace{0.4in}
\end{frame}


\begin{frame}
  \frametitle{ZK + bitcoin: Zero-Knowledge contingent payments \small{(by Greg Maxwell)}}
\begin{enumerate}
 \item Alice chooses cryptographic key  $K$, sends $h =HASH(K)$.\pause
 \item Alice sends encrypted solution $C=E_K(S)$ to Bob; and proves in ZK: ``C is encryption of sudoku solution under key who's hash is $h$.\pause
  \item Bob makes bitcoin ``hash-locked-transaction'' to Alice with $h$.\pause
 \item Alice reveals $K$ to unlock her funds.\pause
 \item Bob can now use $K$ to decrypt solution.\pause
\end{enumerate}

\end{frame}


\begin{frame}
  \frametitle{More on the mathy side: Schnorr's discrete log protocol}
Given $g^x$, prove you know $x$ without revealing it.
\end{frame}
\begin{frame}
  \frametitle{More on the mathy side: Schnorr's discrete log protocol}
Given $X\defeq g^x$, prove you know $x$ without revealing it.
\vspace{0.4in}
\begin{enumerate}
 \item Prover chooses random $r$, sends $R\defeq g^r$.\pause
 \item Verifier chooses random $c$\pause
 \item Prover sends $u\defeq x\cdot c +r$\pause
\item Verifier checks $X\cdot R = g^u$. 
\end{enumerate}

\end{frame}
\begin{frame}
 \frametitle{In parallel to ZK, a  big breakthrough: succinct verification}
 \begin{itemize}
  \item  1990, sumcheck - ``Can prove a sudoku \emph{doesn't} have a solution without verifier going through all options''\pause
  \item 1998, PCP theorem - ``The proof that a sudoku puzzle has a solution can be encoded such that the verifier only needs to read three bits'' 
 \end{itemize}

\end{frame}

\begin{frame}
 \frametitle{Zero-knowledge and succinctness - a love story}
\begin{itemize}
 \item Succinct verification+merkle trees $\rightarrow$ small proofs
 \item When the proof is small easier to make it zk - less places information can hide.
\end{itemize}
 
 \begin{figure}
  \includegraphics[width=150pt]{kilian.png}
\end{figure}
\end{frame}

\begin{frame}
 \frametitle{Succinct arguments in a nutshell}   % Insert frame title between curly braces
 Public program $T$, public output $z$.\\ \pause
 \vspace{0.4in}
 Want to prove ``I know input $x$ for program $T$ that generates output $z$.\\ \pause
 \vspace{0.4in}
Want proof size and verification time to be much smaller than run time of $T$. \\ 
{\small (SNARK:=Succinct Non-Interactive Argument of Knowledge)}\\ \pause
 \vspace{0.4in}
 Arithmetization {\small [LFKN,......]}: Reduce claim to claim of form ''I know polynomials that satisfy some identity`` \pause
\end{frame}
\begin{frame}
 \frametitle{Succinct arguments in a nutshell}   % Insert frame title between curly braces
 Advantage of claims about polynomials is that suffice to check at one random point \\ \pause
 \vspace{0.4in}
 But need to solve ''chicken and egg problem``: Prover must commit to polynomials before knowing the challenge point. 
 \vspace{0.4in}
 
\end{frame}
\begin{frame}
 \frametitle{Polynomial commitment schemes {\small [KZG, 10]}}   % Insert frame title between curly braces
\begin{itemize}
 \item Prover send short commitment $\cm(f)$ to polynomial.\pause
 \item Later Verifier can choose value $i\in \F$.\pause
 \item Prover sends back $z=f(i)$ ; together with proof $\open{f,i}$ that $z$ is correct.\pause
\end{itemize}
KZG give us PCS with commitments and openings are practically 32 bytes.\\ 
Notation: $\enc{x}=g^x$ where $g$ generator of elliptic curve group.
\end{frame}


\begin{frame}
 \frametitle{Elliptic curve pairings - some serious math magic}
 Groups $G,G_t$
 such that
 \begin{itemize}
  \item $G$ is an EC with hard discrete log - from $g^x$ hard to find $x$, for generator $g\in G$.\pause %/- written *additively* - $x\cdot g$ , instead of $g^x$.
  \item We have a map $e:G:\rightarrow G_t$ such that
  \[e( g^a, g^b) = g_t^{a\cdot b}\]
  
 \end{itemize}

\end{frame}

\begin{frame}
Notation: $\enc{x}=g^x$ where $g$ generator of elliptic curve group.
\end{frame}


\begin{frame}
\frametitle{The KZG polynomial commitment scheme}
 Setup: $\enc{1},\enc{x},\ldots,\enc{x^d}$, for random $x\in \F$.\\ \pause
 \vspace{0.4in}
 $\cm(f)\defeq   \enc{f(x)}$\\ \pause
 \vspace{0.4in}
$\open{f,i}\defeq \enc{h(x)}$, where
 $h(X)\defeq \frac{f(X)-f(i)}{X-i}$\\ \pause
 \vspace{0.4in}
 $\verify{\cm,\pi,z,i}:$
\[e(\cm-\enc{z},\enc{1}) \stackrel{?}{=} e(\pi, \enc{x-i})\]
\end{frame}
\begin{frame}
\frametitle{Multiset equality check}
Given $a,b\in \F^3$, want to check $\{b_1,b_2,b_3\} \stackrel{?}{=} \{a_1,a_2,a_3\}$ \\ \pause
 \vspace{0.2in}

  
 Choose random $\gamma\in \F$. Check
  \[(a_1 + \gamma)(a_2+ \gamma)(a_3 + \gamma) \stackrel{?}{=} (b_1+\gamma)(b_2+\gamma)(b_3+\gamma)\]\pause
%   \[a'_1 = a_1 + \gamma, a'_2 = a_2 + \gamma, a'_3 = a_3 + \gamma\]
%  \[b'_1 = b_1 + \gamma, b'_2 = b_2 + \gamma, b'_3 = b_3 + \gamma\]\pause

 \vspace{0.2in}
 If $a,b$ different as sets then w.h.p products different.\pause
 
\end{frame}



\begin{frame}
\frametitle{Multiset equality check - polynomial version}
Given $f,g\in \polysofdeg{d}$, want to check $\sett{f(x)}{x\in H} \stackrel{?}{=} \sett{g(x)}{x\in H}$ as multisets \\ 
 
\end{frame}
\begin{frame}
\frametitle{Reduces to:}   % Insert frame title between curly braces
 $H=\set{\gen,\gen^2,\ldots,\gen^n}$.\\
 \vspace{0.2in}
 
 $\prv$ has sent $f',g'\in \polysofdeg{n}$.\\
 \vspace{0.2in}
 Wants to prove:
 \[\prod_{i\in [n]}  f(\gen^i) = \prod_{i\in [n]} g(\gen^i)\]

 $f\defeq f' +\gamma,g\defeq g'+\gamma$

\end{frame}

\begin{frame}
\frametitle{Multiplicative subgroups:}   % Insert frame title between curly braces
 $H=\set{\gen,\gen^2,\ldots,\gen^n=1}$.\\
 \vspace{0.2in}
 $L_i$ is i'th lagrange poly of $H$:
 \[L_i(\alpha^i)=1,L_i(\alpha^j) =0, j\neq i\]
\end{frame}


\begin{frame}
\frametitle{Checking products with $H$-ranged protocols \small{[GWC19]}}   % Insert frame title between curly braces
 \begin{enumerate}
  \item $\prv$ computes $Z$ with 
  $ Z(\gen)=1, Z(\gen^i) = \prod_{j<i}  f(\gen^j)/g(\gen^j)$.
  \item Sends $Z$ to $\ideal$.   \pause
  \item $\ver$ checks following identities on $H$.
  \begin{enumerate}
   \item $L_1(X) (Z(X)-1) =0$
   \item $Z(X) f(X) = Z(\gen\cdot X)g(X)$\pause
 \end{enumerate}

 \end{enumerate}
 \vspace{0.2in}
% We get $\aggdeg{\prot}=n+2n -|H| = 2n$.


\end{frame}

% \begin{frame}
% 
%  $\cm(f)\defeq   \enc{f(x)}$\\
%  \vspace{0.4in}
% $\open{f,i}\defeq \enc{h(x)}$, where
%  $h(X)\defeq \frac{f(X)-f(i)}{X-i}$\\
%  \vspace{0.4in}
%  $\verify{\cm,\pi,z,i}:$
% \[e(\cm-\enc{z},\enc{1}) \stackrel{?}{=} e(\pi, \enc{x-i})\]
% \end{frame}
% 
% 
%  
% 
% 
% \begin{frame}
% 
%  $\cm(f)\defeq   \enc{f(x)}$\\
%  \vspace{0.4in}
% $\open{f,i}\defeq \enc{h(x)}$, where
%  $h(X)\defeq \frac{f(X)-f(i)}{X-i}$\\
%  \vspace{0.4in}
%  $\verify{\cm,\pi,z,i}:$
% \[e(\cm-\enc{z},\enc{1}) \stackrel{?}{=} e(\pi, \enc{x-i})\]
%  \vspace{0.4in}
% \textbf{Thm}{\footnotesize{[KZG,MBKM]}}: \emph{This works in the Algebraic Group Model.}
%  \end{frame}
% 
% 

\end{document}
