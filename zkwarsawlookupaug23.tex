\documentclass[shadesubsections,compress,14pt,mathserif]{beamer}
\usepackage[danish]{babel}	
\usepackage{tikz}
\usetikzlibrary{shapes, positioning}
\usenavigationsymbolstemplate{}
% \usepackage{pgfplots}
\usepackage[absolute,overlay]{textpos}
\usepackage{amsthm,amsfonts,amsmath}
%\usepackage[T1]{fontenc}
% \usepackage{fullpage}
% Dokumentets sprog
%\usepackage{mathtools}
%\usepackage{pxfonts}
\usepackage{eulervm}
 \usepackage{numdef,graphicx}
\usepackage{framed} 

% Class options include: notes, notesonly, handout, trans,
%                        hidesubsections, shadesubsections,
%                        inrow, blue, red, grey, brown

% Theme for beamer presentation.
%\usepackage{beamertheme} 
% Other themes include: beamerthemebars, beamerthemelined, 
%                       beamerthemetree, beamerthemetreebars  
%  \usepackage[latexcolors]
\everymath{\color{purple}}
\newcommand{\adv}{\ensuremath{\mathcal A}}
\newcommand{\F}{\ensuremath{{\mathbb F}}}
\newcommand{\Z}{\ensuremath{{\mathbb Z}}\xspace}
\newcommand{\Fclosure}{\ensuremath{{\overline{\mathbb{F}}}_p}}
\newcommand{\set}[1]{\ensuremath{\left\{#1\right\}}}
\newcommand{\sett}[2]{\ensuremath{\left\{#1\right\}_{#2}}}
% \newcommand{\enc}[1]{\ensuremath{\left[#1\right ]}}
\newcommand{\enc}[1]{\ensuremath{#1\cdot G}}
\newcommand{\kzg}[1]{\ensuremath{\enc{#1(x)}}}
\newcommand{\cm}{\ensuremath{\mathsf{cm}}}
\newcommand{\open}[1]{\ensuremath{\mathsf{open}(#1)}}
\newcommand{\verify}[1]{\ensuremath{\mathsf{verify}(#1)}}
\newcommand{\defeq}{\ensuremath{:=}}
\newcommand{\helper}{\ensuremath{\mathcal{H}}}
\newcommand{\ver}{\ensuremath{\mathcal{V}}}
\newcommand{\prv}{\ensuremath{\mathcal{P}}}
 \newcommand{\polysofdeg}[1]{\F_{< #1}[X]}
%  \newcommand{\endoss}{\ensuremath{\mathrm{END}_E}}
 \newcommand{\hl}[1]{\textbf{\textit{#1}}}
 \newcommand{\polys}{\F[X]}
\newcommand{\acc}{{\mathbf{acc}}}
\newcommand{\ideal}{\mathbf{I}}
\newcommand{\gen}{\alpha}
\newcommand{\spac}{\\  \vspace{0.2in} \noindent}
\newcommand{\polylog}{\ensuremath{\mathsf{polylog}}\xspace}
% \renewcommand{\bf}{\begin{frame}}
% \newcommand{\ef}{\end{frame}}
%\setbeamersize{text margin left=3mm,text margin right=3mm}  
\newcommand{\nl}{\\ \pause \vspace{0.2in}}
\newcommand{\nlnp}{\\ \vspace{0.2in}}
\newcommand{\stitle}[1]{{\large{\textcolor{purple}{\emph{#1}}}}}
\DeclareMathAlphabet{\mathpgoth}{OT1}{pgoth}{m}{n}	
\newcommand{\cq}{\mathpgoth{cq} }
\newcommand{\cqstar}{\ensuremath{\mathpgoth{cq^{\mathbf{*}} }}}
\newcommand{\flookup}{{\mathsf{\mathpgoth{Flookup}}}}
\newcommand{\baloo}{{\mathrm{ba}\mathit{loo}}}
\newcommand{\caulkp}{{\mathsf{\mathrel{Caulk}\mathrel{\scriptstyle{+}}}}}
\newcommand{\caulk}{{\mathsf{Caulk}}}
\newcommand{\plookup}{\ensuremath{\mathpgoth{plookup}}\xspace}
\newcommand{\srs}{\ensuremath{\mathsf{srs}}}
\newcommand{\tablegroup}{\ensuremath{\mathbb{H}}\xspace}
\newcommand{\V}{\ensuremath{\mathbf{V} }\xspace}
\newcommand{\bigspace}{\ensuremath{\mathbb{V}}}
\newcommand{\papertitle}{Cached quotients and lookups}
%\newcommand{\authorname}}
\newcommand{\company}{}
\newcommand{\witsize}{{n}}
\newcommand{\witruntime}{\ensuremath{\witsize\log\witsize}\xspace}
\newcommand{\tabsize}{{N}}
\newcommand{\tabruntime}{\ensuremath{\tabsize\log\tabsize}\xspace}
\newcommand{\Gone}{{\mathbb G}_1}
\newcommand{\Gi}{\ensuremath{{\mathbb G}_i}\xspace}
\title{ \bf \papertitle \\[0.72cm]}
\author{Ariel Gabizon \\
Zeta Function Technologies} 

%\usefonttheme{professionalfonts}
%\usefonttheme[onlymath]{serif}
\begin{document}
\boldmath
% Creates title page of slide show using above information
\begin{frame}
  \titlepage
\end{frame}
                           % typeset with the notes or notesonly class options
\begin{frame}
\frametitle{Constraints vs Lookups}
\textbf{Example:} Check $0\leq x \leq 2^n-1$\nl
\textbf{Constraint approach:}\\
Prover sends $b_0,\ldots,b_{n-1}$. Shows:
\begin{itemize}
 \item 
$\forall i, b_i\in \set{0,1}$
\item
$\sum_{i} b_i2^i =x$.\nl
\end{itemize}
Requires $n+1$ constraints.

\end{frame}
\begin{frame}
\frametitle{Lookup approach}
Preprocess table $T=\set{0,\ldots,2^n-1}$.Let $N:=|T|$. \\ \noindent 
Devise protocol to check $x\in T$.\nl
\textit{\color{blue}{Old results - good when amortized:}}\\ \noindent
\noindent
\textbf{Thm[plookup]:}  Can check $m$ different $x$'s are in $T$ in $O(m+ N)$
 constraints.\nl

\textit{\color{blue}{New results - prover doesn't pay for table size!!\\}}
\textbf{Thm [Caulk...$\cq$]:} After $O(N\log N)$ preprocessing, can check $x\in T$, in $O(1)$ constraints. 
\end{frame}
%\section[Outline]{}

% Creates table of contents slide incorporating
% all \section and \subsection commands
% \begin{frame}
%   \tableofcontents
% \end{frame}
% \begin{frame}
% \frametitle{Outline}
%  
% \begin{itemize}
%  \item 
% PCS/KZG review\pause
% \item
% KZG shenanigans 
% \begin{enumerate}
%  \item Committing to sparse polys.
% \item ``cached quotients''\pause
% 
% \end{enumerate}
% \item Lookups
% \begin{enumerate}
%  \item Motivation
% \item  New results:$\cq$
% \end{enumerate}
% \end{itemize}
% \end{frame}
% 
\begin{frame}
{\large{\textbf{Rest of talk:} explain main technical component of new works - \emph{cached quotients}}}\pause
\emph{First - a brief recap of polynomial commitment schemes..}

 \end{frame}
\begin{frame}
 \frametitle{The KZG Polynomial commitment scheme}   % Insert frame title between curly braces
 $G$ - generator of pairing friendly elliptic curve group.\\
 \vspace{0.2in}
 $\srs \defeq \enc{1},\enc{x},\ldots,\enc{x^d}$, for random $x\in \F$.\nl
 For $f\in \F[X]$ of degree $d$: 
$$\color{purple} \cm(f)\defeq   \enc{f(x)}$$\nl
 \textbf{Central Feature:}
 Given $\cm(f)$ and any $a\in \F$; there is short proof for correctness of $z=f(a)$. 
\end{frame}
\begin{frame}
 \frametitle{The KZG Polynomial commitment scheme}   % Insert frame title between curly braces
 $\srs \defeq \enc{1},\enc{x},\ldots,\enc{x^d}$, for random $x\in \F$.\\ 
 $\cm(f)\defeq   \enc{f(x)}$\\ 
 \vspace{0.2in}
 Nice features:\pause
 \begin{itemize}
  \item \textbf{Linearity:} $\cm(f+g) = \cm(f)+\cm(g)$\pause
  \item \textbf{Product checks:} Given $\cm(f_1),\cm(f_2),\cm(g_1),\cm(g_2)$ can check $f_1(X)f_2(X)\stackrel{?}{\equiv} g_1(X)g_2(X)$ via pairings.\\
  (Secure in the Algebraic Group Model)
 \end{itemize}

\end{frame}
\begin{frame}
 \frametitle{Cached Quotients - a motivating example from $\caulk$[ZBKMN]}\pause
 $Z_T(X)=\prod_{a\in T} (X-a)$ 
 a vanishing polynomial of a subset $T\subset \F$ .\nl
 $\cm(Z_T),\cm(f)$ given to verifier.\pause \\
 Prover wants to show $f=Z_S$ for some $S\subset T$.\nl
 
 Can we do this in $O(|S|)$ prover operations?(think $|S|<<|T|$)
\end{frame}
\begin{frame}
\frametitle{Cached quotients idea:}
 The quotient $Z_{T\setminus S}(X) =\frac{Z_T(X)}{Z_S(X)}$ is a ``witness'' to $S\subset T$. \nl
\begin{itemize}
 \item Enough to compute \textbf{commitment} to $Z_{T\setminus S}$. \pause
 \item This commitment is a \textbf{sparse combination} of commitments we can \textbf{precompute}.
\end{itemize}
\emph{details in next slide..}
\end{frame}

\begin{frame}
% \frametitle{Fractional decomposition:}
 For each $i\in T$, let $g_i(X)\defeq Z_{T\setminus\set{i}}(X)$.\nl
 We have {\small[Tomescu et. al]}
 \[Z_{T\setminus S} (X) = \sum_{i\in S} c_i\cdot g_i(X)\]
 for some $c_i\in \F$.\nl
 

 We precompute $\cm(Z_T),\sett{\cm(g_i)}{i\in T}$.\nlnp
\end{frame} 
 \begin{frame}
 Prover then computes in $|S|$ operations:
 \[\color{purple}\pi:=\cm(Z_{T\setminus S}) = \sum_{i\in S} c_i\cdot \cm(g_i)\]\nl
Verifier checks with pairing that:
\[\color{purple}e(\cm(f),\pi) =e(\cm(Z_T),\enc{1})\]
\end{frame}

\begin{frame}
\frametitle{Historical perspective: A trilogy of pairing-based SNARKs}\pause

\begin{enumerate}
 \item \textbf{A new hope (for SNARKs, not the universe)} - [Groth10,\textbf{GGPR},...,Groth16]\pause 
 \vspace{0.2in}
 \item  \textbf{The polynomial commitment scheme strikes back} - [vsql,\textbf{Sonic},Plonk,Marlin,...]\pause 
 \vspace{0.2in}
 
 \item  \textbf{Return of the pairing} - [Caulk,...,cq,..]
 
\end{enumerate}

\end{frame}
\begin{frame}
\frametitle{Other Application in Chapter 3: lincheck }
Fixed $n\times n$ matrix $M$.\nl
Prover has poly $f\in \polysofdeg{n}$.
Verifier $\cm(f)$.
$a\defeq f|_H$ for subgroup $H$ of size $n$.\nl

\textbf{cq-lin:} After preprocessing of $M$, prover can show $M\cdot a =0$
\emph{in $O(n)$ operations.}\nl



\end{frame}

\end{document}
