\documentclass[shadesubsections,compress,14pt,mathserif]{beamer}
\usepackage[danish]{babel}	
\usepackage{tikz,circuitikz}
\usetikzlibrary{shapes, positioning}
\usenavigationsymbolstemplate{}
\usepackage{pgfplots}
\usepackage[absolute,overlay]{textpos}
\usepackage{amsthm,amsfonts}
%\usepackage[T1]{fontenc}
% \usepackage{fullpage}
% Dokumentets sprog
%\usepackage{mathtools}
%\usepackage{pxfonts}
\usepackage{eulervm}
\usepackage[export]{adjustbox}
\everymath{\color{purple}}
% Class options include: notes, notesonly, handout, trans,
%                        hidesubsections, shadesubsections,
%                        inrow, blue, red, grey, brown

% Theme for beamer presentation.
%\usepackage{beamertheme} 
% Other themes include: beamerthemebars, beamerthemelined, 
%                       beamerthemetree, beamerthemetreebars  
\newcommand{\adv}{\ensuremath{\mathcal A}}
\newcommand{\F}{\ensuremath{{\mathbb F}}}
\renewcommand{\P}{\ensuremath{{\mathbb P}}}
\newcommand{\Z}{\ensuremath{{\mathbb Z}}\xspace}
\newcommand{\Fclosure}{\ensuremath{{\overline{\mathbb{F}}}_p}}
\newcommand{\set}[1]{\ensuremath{\left\{#1\right\}}}
\newcommand{\bin}{\ensuremath{\set{0,1}}}
\newcommand{\cube}{\ensuremath{\bin^n}}

\newcommand{\sett}[2]{\ensuremath{\left\{#1\right\}_{#2}}}
\newcommand{\enc}[1]{\ensuremath{\left[#1\right ]}}
% \newcommand{\kzg}[1]{\ensuremath{\enc{#1(x)}}}
\newcommand{\cm}{\ensuremath{\mathsf{cm}}}
\newcommand{\kzg}[1]{\cm(#1)}
\newcommand{\open}[1]{\ensuremath{\mathsf{open}(#1)}}
\newcommand{\verify}[1]{\ensuremath{\mathsf{verify}(#1)}}
\newcommand{\defeq}{\ensuremath{:=}}
\newcommand{\helper}{\ensuremath{\mathcal{H}}}
\newcommand{\ver}{\ensuremath{\mathcal{V}}}
\newcommand{\prv}{\ensuremath{\mathcal{P}}}
 \newcommand{\polysofdeg}[1]{\F_{< #1}[X]}
%  \newcommand{\endoss}{\ensuremath{\mathrm{END}_E}}
 \newcommand{\hl}[1]{\textbf{\textit{#1}}}
 \newcommand{\polys}{\F[X]}
\newcommand{\acc}{{\mathbf{acc}}}
\newcommand{\ideal}{\mathbf{I}}
\newcommand{\gen}{\alpha}
\newcommand{\spac}{\\  \vspace{0.2in} \noindent}
\newcommand{\polylog}{\ensuremath{\mathsf{polylog}}\xspace}
% \renewcommand{\bf}{\begin{frame}}
% \newcommand{\ef}{\end{frame}}
%\setbeamersize{text margin left=3mm,text margin right=3mm}  
\newcommand{\nl}{\\ \pause \vspace{0.2in}}
\newcommand{\nlnp}{\\ \vspace{0.2in}}
\newcommand{\stitle}[1]{{\large{\textcolor{purple}{\emph{#1}}}}}
\DeclareMathAlphabet{\mathpgoth}{OT1}{pgoth}{m}{n}	
\newcommand{\cq}{\mathpgoth{cq} }
\newcommand{\cqstar}{\ensuremath{\mathpgoth{cq^{\mathbf{*}} }}\xspace}
\newcommand{\flookup}{\ensuremath{\mathsf{\mathpgoth{Flookup}}}\xspace}
\newcommand{\baloo}{\ensuremath{\mathrm{ba}\mathit{loo}}\xspace}
% \newcommand{\caulkp}{\ensuremath{\mathsf{\mathrel{Caulk}\mathrel{\scriptstyle{+}}}}\xspace}
\newcommand{\caulk}{\ensuremath{\mathsf{Caulk}}\xspace}
\newcommand{\plookup}{\ensuremath{\mathpgoth{plookup}}\xspace}
\newcommand{\srs}{\ensuremath{\mathsf{srs}}}
\newcommand{\tablegroup}{\ensuremath{\mathbb{H}}\xspace}
\newcommand{\V}{\ensuremath{\mathbf{V} }\xspace}
% \newcommand{\caulk}{{\mathsf{Caulk}}}
% \newcommand{\caulkp}{{\mathsf{\mathrel{Caulk}\mathrel{\scriptstyle{+}}}}}
\newcommand{\bigspace}{\ensuremath{\mathbb{V}}}

%\setbeamersize{text margin left=3mm,text margin right=3mm} 
\title{\large{GFFT on projective line}}    % Enter your title between curly braces
\author{\small{Ariel Gabizon}\\                 % Enter your name between curly braces
\tt{\footnotesize{Zeta Function Technologies}                                       } }      % Enter your institute name between curly braces
\date{}                    % Enter the date or \today between curly braces
%\usefonttheme{professionalfonts}
%\usefonttheme[onlymath]{serif}
\begin{document}
\boldmath
% Creates title page of slide show using above information
\begin{frame}
  \titlepage
\end{frame}


% \begin{frame}
% \large{plonk is a protocol to make short proofs about circuit satisfiability.}
%  \begin{figure}
%   \includegraphics[width=260pt]{circuit.png}
% \end{figure}
% \end{frame}
\begin{frame}
 \frametitle{FFT Reminder}
 
 
 
 $S=\set{g,g^2,\ldots,g^{n}=1},n=2^k$\nl
 Want to evaluate $f(X)\in \F[X]$ of deg $<n$ on $S$.\pause
 Use recursive formula
 \[f(X)=f_e(X^2)+X\cdot f_o(X^2)\]
 
 Since the map $x\to x^2$ is 2-to-1 on $S$, this reduces $n$ evals of $f$ to $n/2$ evals of two deg $n/2$ polys.\nl

 
 Requires $n|p-1$, where $p=|\F|$.
 
 Can we do something when  $n|(p+1)$ instead?? 
 % \begin{itemize}
%  \item 
% \end{itemize}
 
\end{frame}
\begin{frame}
 \frametitle{Reflection - why does FFT work?}
 $S=\set{g,g^2,\ldots,g^{n}=1},n=2^k$\nl
The map $\sigma(x)=g\cdot x$ goes over $S$ as a cycle.\nl
\begin{itemize}
\item Let $\tau = \sigma^{n/2}$. So $\tau(x)=-x$, and $\tau^2(x)=x$.
\item The elements of $S$ split into disjoint pairs $(a,-a)$.\pause
\item Define $N(X)=X \cdot \tau(X)=-X^2$. $N$ maps elements of a pair to the same output.\nl  
\textit{Can we find a set of size $p+1$ with a similar cyclical $\sigma$?}
\end{itemize}

\end{frame}
\begin{frame}
 \frametitle{The Projective line and fractional transformations}
 
 How to get set $\P$ of size $2^k$?
 Look at \emph{projective line} $\P\defeq \F\cup\infty$\nl
 Take fractional map:
 $\sigma(x)= \frac{1}{ax+b}$
 
 Define: $\sigma(-b/a)=\infty,\sigma(\infty)=0$.\nl
 
 \textbf{claim:}For the right choice of $a,b$ $\sigma$ makes a cycle over all of \P!
 
 
 
\end{frame}
\begin{frame}
 \frametitle{How do people formally represent the projective line?}
 \textbf{Projective coordinates:}
 Represent $a\in F$ by  $(c,d)$ with $a=c/d$ e.g. $(a,1)$.\nlnp
 So $\infty=(1,0)$.\nl
 

\end{frame}

\begin{frame}
 \frametitle{How do people formally represent the projective line?}
\textbf{As a circle in the plane:}\nlnp
\begin{circuitikz}
    % Draw the real line
    \draw[thick] (-2,0) -- (2,0);
    \node at (2.5,0) {};
    
    % Draw the larger circle a bit lower above the real line
    \draw (0,1.5) node[circle,draw,minimum size=1.5cm] (circle) {};
    \node at (0,1.5) {};
\filldraw (circle.north) circle (2pt) node[above] {$\infty$};
    % Draw a line from the top of the circle to somewhere on the real line
    \draw (circle.north) -- (1,0);

    % Add little black dots at the connection points
    \filldraw (circle.north) circle (2pt);
    \filldraw (1,0) circle (2pt);

    % Connect the circle to the real line with a dashed line
    \draw[dashed] (0,0) -- (circle);

    % Calculate the intersection point of the line and the circle
    \coordinate (intersection) at ($(circle) + (-41:0.75cm)$);
    \filldraw (intersection) circle (2pt);
\end{circuitikz}
\vspace{0.3in}\nlnp
\textit{See Circle STARK paper [HLP24] for this approach}
\end{frame}
\begin{frame}
 \frametitle{How do people formally represent the projective line?}
\textbf{As places of the field $K=\F(X)$:}\nl
{\textcolor{green} {Dfn:}} A \emph{valuation ring} of $R\subset \F(X)$ is a subring such that
$\forall y\in K$ $y\in R$ or $1/y\in R$ (or both).\nl
% \textit{\small{ Reminder:ring is set of elements closed under add/mult but maybe no divide}}\nl

\textbf{Example:}  Choose $a\in F$, take $R_a=\{f(X)/g(X)| f,g\in \F[X], g(a)\neq 0 \}$.\nl

% \textit{Reminder: an ideal $I\subset R$ is closed under addition; and multiplication by \emph{any element of $R$}}

% The unique maximal ideal of $R_a$ is $I_a=\set{(X-a)\cdot r|r\in R}$


% \textit{Cool thing: $R_a/I_a=\F$. And we can evaluate $r\in R$ at $a$ by taking $r\;\mod\; I$}\nl


Valuation rings in $K$, are also called ``places'' of $K$. 


\end{frame}
\begin{frame}
The unique maximal ideal of $R_a$ is $I_a=\set{(X-a)\cdot r|r\in R}$\nl
\textit{Cool thing: $R_a/I_a=\F$. And we can evaluate $r\in R_a$ at $a$ by taking $r\;\mod\; I_a$}\nl
This gives the same result as ``normal'' evaluation!
 
\end{frame}

\begin{frame}
 \frametitle{The infinity point in the algebraic representation}
 There is \emph{one more} place of degree one in $K$:
 $R_{\infty}=\set{f(X)/g(X)|deg(f)\leq deg(g)}$. \nl
 $R_{\infty}=$ ``the set of functions that can be evaluated at infinity''
 
\end{frame}
\begin{frame}
 \textit{What does this last part have to do with FFT?}\nlnp
Like in regular FFT - we'll end up needing to represent $f(X)$ as a combination of two functions $f_e(N(X)), f_o(N(X))$
of half the ``degree''; where $N$ will be a degree two rational function.\nl
Working within the function field gives us convenient tools to construct the right bases for representing $f,f_e,f_o$,
and defining degree in the right way.

 
\end{frame}


\end{document}
