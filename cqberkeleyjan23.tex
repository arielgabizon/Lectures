\documentclass[shadesubsections,compress,14pt,mathserif]{beamer}
\usepackage[danish]{babel}	
\usepackage{tikz}
\usetikzlibrary{shapes, positioning}
\usenavigationsymbolstemplate{}
\usepackage{pgfplots}
\usepackage[absolute,overlay]{textpos}
\usepackage{amsthm,amsfonts}
%\usepackage[T1]{fontenc}
% \usepackage{fullpage}
% Dokumentets sprog
%\usepackage{mathtools}
%\usepackage{pxfonts}
\usepackage{eulervm}

% Class options include: notes, notesonly, handout, trans,
%                        hidesubsections, shadesubsections,
%                        inrow, blue, red, grey, brown

% Theme for beamer presentation.
%\usepackage{beamertheme} 
% Other themes include: beamerthemebars, beamerthemelined, 
%                       beamerthemetree, beamerthemetreebars  
\newcommand{\adv}{\ensuremath{\mathcal A}}
\newcommand{\F}{\ensuremath{{\mathbb F}}}
\newcommand{\Z}{\ensuremath{{\mathbb Z}}\xspace}
\newcommand{\Fclosure}{\ensuremath{{\overline{\mathbb{F}}}_p}}
\newcommand{\set}[1]{\ensuremath{\left\{#1\right\}}}
\newcommand{\sett}[2]{\ensuremath{\left\{#1\right\}_{#2}}}
\newcommand{\enc}[1]{\ensuremath{\left[#1\right ]}}
\newcommand{\kzg}[1]{\ensuremath{\enc{#1(x)}}}
\newcommand{\cm}{\ensuremath{\mathsf{cm}}}
\newcommand{\open}[1]{\ensuremath{\mathsf{open}(#1)}}
\newcommand{\verify}[1]{\ensuremath{\mathsf{verify}(#1)}}
\newcommand{\defeq}{\ensuremath{:=}}
\newcommand{\helper}{\ensuremath{\mathcal{H}}}
\newcommand{\ver}{\ensuremath{\mathcal{V}}}
\newcommand{\prv}{\ensuremath{\mathcal{P}}}
 \newcommand{\polysofdeg}[1]{\F_{< #1}[X]}
%  \newcommand{\endoss}{\ensuremath{\mathrm{END}_E}}
 \newcommand{\hl}[1]{\textbf{\textit{#1}}}
 \newcommand{\polys}{\F[X]}
\newcommand{\acc}{{\mathbf{acc}}}
\newcommand{\ideal}{\mathbf{I}}
\newcommand{\gen}{\alpha}
\newcommand{\spac}{\\  \vspace{0.2in} \noindent}
\newcommand{\polylog}{\ensuremath{\mathsf{polylog}}\xspace}
% \renewcommand{\bf}{\begin{frame}}
% \newcommand{\ef}{\end{frame}}
%\setbeamersize{text margin left=3mm,text margin right=3mm}  
\newcommand{\nl}{\\ \pause \vspace{0.2in}}
\newcommand{\nlnp}{\\ \vspace{0.2in}}
\newcommand{\stitle}[1]{{\large{\textcolor{purple}{\emph{#1}}}}}
\DeclareMathAlphabet{\mathpgoth}{OT1}{pgoth}{m}{n}	
\newcommand{\cq}{\mathpgoth{cq} }
\newcommand{\cqstar}{\ensuremath{\mathpgoth{cq^{\mathbf{*}} }}\xspace}
\newcommand{\flookup}{\ensuremath{\mathsf{\mathpgoth{Flookup}}}\xspace}
\newcommand{\baloo}{\ensuremath{\mathrm{ba}\mathit{loo}}\xspace}
\newcommand{\caulkp}{\ensuremath{\mathsf{\mathrel{Caulk}\mathrel{\scriptstyle{+}}}}\xspace}
\newcommand{\caulk}{\ensuremath{\mathsf{Caulk}}\xspace}
\newcommand{\plookup}{\ensuremath{\mathpgoth{plookup}}\xspace}
\newcommand{\srs}{\ensuremath{\mathsf{srs}}}
\newcommand{\tablegroup}{\ensuremath{\mathbb{H}}\xspace}
\newcommand{\V}{\ensuremath{\mathbf{V} }\xspace}
\newcommand{\bigspace}{\ensuremath{\mathbb{V}}}
\newcommand{\papertitle}{$\mathfrak{cq}$: Cached quotients for fast lookups}
%\newcommand{\authorname}}
\newcommand{\company}{}
\title{ \bf \papertitle \\[0.72cm]}
\author{ Liam Eagen \and Dario Fiore \and \textbf{Ariel Gabizon} } 

%\usefonttheme{professionalfonts}
%\usefonttheme[onlymath]{serif}
\begin{document}
\boldmath
% Creates title page of slide show using above information
\begin{frame}
  \titlepage
\end{frame}
                           % typeset with the notes or notesonly class options

%\section[Outline]{}

% Creates table of contents slide incorporating
% all \section and \subsection commands
\begin{frame}
  \tableofcontents
\end{frame}
\begin{frame}
\frametitle{Outline}
 
\begin{itemize}
 \item 
PCS/KZG review\pause
\item
KZG shenanigans 
\begin{enumerate}
 \item Committing to sparse polys.
\item ``cached quotients''\pause

\end{enumerate}
\item Lookups
\begin{enumerate}
 \item Motivation
\item log derivative protocol[Eagen, Hab\"ock,..]
\item  $\cq$
\end{enumerate}
\end{itemize}
\end{frame}

\begin{frame}
 \frametitle{Polynomial commitment schemes {\small [KZG, 10]}}   % Insert frame title between curly braces
\begin{itemize}
 \item Prover send short commitment $\cm(f)$ to polynomial.\pause
 \item Later Verifier can choose value $i\in \F$.\pause
 \item Prover sends back $z=f(i)$ ; together with proof $\open{f,i}$ that $z$ is correct.\pause
\end{itemize}
KZG give us PCS with commitments and openings are practically 32-48 bytes.\\ 
Notation: $\enc{x}=x\cdot g$ where $g$ generator of (an additive) elliptic curve group.
\end{frame}
\begin{frame}
\frametitle{KZG10:}
 $\srs \defeq \enc{1},\enc{x},\ldots,\enc{x^d}$, for random $x\in \F$.\\ \pause
 \vspace{0.4in}
 $\cm(f)\defeq   \enc{f(x)}$\\ \pause
 \vspace{0.4in}
$\open{f,i}\defeq \enc{h(x)}$, where
 $h(X)\defeq \frac{f(X)-f(i)}{X-i}$\\ \pause
 \vspace{0.4in}
 $\verify{\cm,\pi,z,i}:$
\[e(\cm-\enc{z},\enc{1}) \stackrel{?}{=} e(\pi, \enc{x-i})\]
\end{frame}

\begin{frame}
 \frametitle{Shenanigan $\#1$: Committing to sparse polys}
\textbf{notation:} parameters $n<<N$, $d\defeq N-1$.
 \\
 \vspace{0.4in}
 $\bigspace=\set{\omega,\ldots,\omega^N}\subset \F$ subgroup of size $N$.\nl
Say $A\in \polysofdeg{N}$  is $n$-sparse if has at most $n$ non-zeroes on $\bigspace$.\nl
For $i\in [N]$, denote $A_i\defeq A(\omega^i)$\\
$L_1(X),\ldots,L_N(X)$ - Lagrange basis of $\bigspace$ - $(L_i)_j=0$ when $i\neq j$.
\end{frame}

\begin{frame}
 \frametitle{Committing to sparse polys}
From $\srs \defeq \enc{1},\enc{x},\ldots,\enc{x^d}$, we can precompute in $O(N\log N)$ operations 
the KZG commitments of $\bigspace$'s Lagrange Base:\\
$\srs_L \defeq \set{\enc{L_1(x)},\ldots,\enc{L_N(x)}}$\nl

Now for $n$-sparse $A(X)$ of degree$<N$ compute 
\[\cm(A)=\enc{A(x)}=\sum_{i\in [N], A_i\neq 0} A_i\cdot \kzg{L_i}
\]

\end{frame}
 
\begin{frame}
 \frametitle{Shenanigan $\#2$: ``Cached quotients'' method}
 \textbf{Scenario:}
$T(X)\in \polysofdeg{N}$ preprocessed poly.
$Z_\bigspace(X)$-vanishing poly of $\bigspace$.\\
Input: $n$-sparse $A(X)\in\polysofdeg{N}$.\nl

$V$ has $\cm(A)$. Want to prove to $V$ that:\\
$Z_V(X)$ divides $A(X)T(X)$ \textit{using $O(n)$ prover operations}.

\end{frame}
\begin{frame}
\frametitle{``Cached quotients'' method}
There exists quotient $Q_A(X)$ such that $A\cdot T\equiv Z_V\cdot Q_A$.\nl

We'll compute $\kzg{Q_A}$ in $O(n)$ operations:\nl


\textbf{preprocessing:}
For each $i\in [N]$, compute $\kzg{Q_i}$ such that for some $R_i(X)\in \polysofdeg{N}$
\[L_i(X)\cdot T(X) = Q_i(X)\cdot Z_\bigspace(X) +R_i(X)\]\nl

Also precompute $\kzg{Z_\bigspace},\kzg{T}$
\end{frame}
\begin{frame}
\frametitle{``Cached quotients'' method}
After preprocessing, prover can compute
\[\kzg{Q_A}=\sum_{i\in [N],A_i\neq 0} A_i \cdot \kzg{Q_i}\]\nl
Verifier can check:
\[e(\kzg{A},\kzg{T})=e(\kzg{Q_A},\kzg{Z_{\bigspace}})\]\nl
In algebraic group model[FKL] can prove this is sound.
\end{frame}

    

\begin{frame}
{\huge{Lookup protocols}}
 \end{frame}







\begin{frame}
\frametitle{Constraints vs Lookups}
\textbf{Example:} Check $0\leq x \leq 2^n-1$\nl
Constraint approach:
$\prv$ sends $x_0,\ldots,x_{n-1}$
Proves
\begin{itemize}
 \item 
$\forall i, x_i\in \set{0,1}$
\item
$\sum_{i} x_i2^i =x$.\nl
\end{itemize}
Requires $n+1$ ``gates''.

\end{frame}
\begin{frame}
\frametitle{Lookup approach}
Preprocess table $T=\set{0,\ldots,2^n-1}$
Devise protocol to check $x\in T$.\nl
\textbf{Thm-informal [Arya..plookup]:} Check can be done in amortized $O(1)$ constraints 
per check, when have $O(|T|)$ checks.\nl

\textbf{Thm [Caulk..$\cq$]:} Can be done in $O(1)$ constraints without need of amortization!
\end{frame}
\begin{frame}
 \frametitle{The question in polynomials:}
 \textbf{Preprocessed:} $\bigspace\subset \F$ subgroup of size $N$. $H\subset \F$ subgroup of size $n$. $T\in \polysofdeg{N}$.\nl
 Input: $f\in \polysofdeg{n}$.  $\cm(f)$ given to $V$.\pause
 
\vspace{0.4in}
 Want to convince $V$ that $f|_H\subset T|_\bigspace$ in $O(n)$ prover operations.
 
\end{frame}
\begin{frame}
\frametitle{Log-derivative approach:}
\textbf{Lemma[Hab\"ock]:}
$f|_H\subset T|_\bigspace$ if and only if there exists $m(X)\in \polysofdeg{N}$ s.t.  as rational functions
\[\sum_{i\in [N]} \frac{m_i}{X+ T_i}=  \sum_{a\in H} \frac{1}{X+f(a)}   \]\nl
\textit{Strategy: check this identity at random $\beta \in \F$.}

\end{frame}
\begin{frame}
\frametitle{$\cq$}
\textbf{Main prover task:}
Compute polynomial $A(X)$ that interpolates RHS on $\bigspace$, and
prove it correct:
\[A_i=\frac{m_i}{\beta+T_i}, \forall i\in [N]\]\nl

Can be done via the ``KZG shenanigans'' we described before.\nl Must compute $\kzg{Q_A}$ where
\[A(X)(\beta+T(X))-m(X)=Q_A(X)Z_\bigspace(X).\]

\end{frame}
\end{document}
