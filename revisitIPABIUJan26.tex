\documentclass[shadesubsections,compress,14pt,mathserif]{beamer}
\usepackage[danish]{babel}	
\usepackage{tikz,circuitikz}
\usepackage[normalem]{ulem}
\usetikzlibrary{shapes, positioning}
\usenavigationsymbolstemplate{}
\usepackage{xcolor,pgfplots,bm}
\usepackage[absolute,overlay]{textpos}
\usepackage{amsthm,amsfonts}
%\usepackage[T1]{fontenc}
% \usepackage{fullpage}
% Dokumentets sprog
%\usepackage{mathtools}
%\usepackage{pxfonts}
\usepackage{eulervm}
\usepackage[export]{adjustbox}
\everymath{\color{purple}}
\definecolor{darkred}{rgb}{0.75, 0, 0.25} 
% Class options include: notes, notesonly, handout, trans,
%                        hidesubsections, shadesubsections,
%                        inrow, blue, red, grey, brown

% Theme for beamer presentation.
%\usepackage{beamertheme} 
% Other themes include: beamerthemebars, beamerthemelined, 
%                       beamerthemetree, beamerthemetreebars  
\newcommand{\minus}{\scalebox{0.5}[1.0]{\( - \)}}
\newcommand{\adv}{\ensuremath{\mathcal A}}
\newcommand{\F}{\ensuremath{{\mathbb F}}}
\newcommand{\G}{\ensuremath{{\mathbb G}}}
\renewcommand{\P}{\ensuremath{{\mathbb P}}}
\newcommand{\Z}{\ensuremath{{\mathbb Z}}\xspace}
\newcommand{\Fclosure}{\ensuremath{{\overline{\mathbb{F}}}_p}}
\newcommand{\set}[1]{\ensuremath{\left\{#1\right\}}}
\newcommand{\bin}[1]{\ensuremath{\set{0,1}^{#1}}}
\newcommand{\cube}{\ensuremath{\bin^n}}

\newcommand{\sett}[2]{\ensuremath{\left\{#1\right\}_{#2}}}
\newcommand{\enc}[1]{\ensuremath{\left[#1\right ]}}
% \newcommand{\kzg}[1]{\ensuremath{\enc{#1(x)}}}
\newcommand{\cm}{\ensuremath{\mathsf{cm}}}
\newcommand{\kzg}[1]{\cm(#1)}
\newcommand{\open}{\ensuremath{\mathsf{open}}}
\newcommand{\verify}[1]{\ensuremath{\mathsf{verify}(#1)}}
\newcommand{\defeq}{\ensuremath{:=}}
\newcommand{\helper}{\ensuremath{\mathcal{H}}}
\newcommand{\ver}{\ensuremath{\mathcal{V}}}
\newcommand{\prv}{\ensuremath{\mathcal{P}}}
 \newcommand{\polysofdeg}[1]{\F_{< #1}[X]}
%  \newcommand{\endoss}{\ensuremath{\mathrm{END}_E}}
 \newcommand{\hl}[1]{\textbf{\textit{#1}}}
 \newcommand{\polys}{\F[X]}
\newcommand{\acc}{{\mathbf{acc}}}
\newcommand{\rej}{{\mathbf{rej}}}
\newcommand{\ideal}{\mathbf{I}}
\newcommand{\gen}{\alpha}
\newcommand{\spac}{\\  \vspace{0.2in} \noindent}
\newcommand{\polylog}{\ensuremath{\mathsf{polylog}}\xspace}
% \renewcommand{\bf}{\begin{frame}}
% \newcommand{\ef}{\end{frame}}
%\setbeamersize{text margin left=3mm,text margin right=3mm}  
\newcommand{\nl}{\\ \pause \vspace{0.2in}}
\newcommand{\nlnp}{\\ \vspace{0.2in}}
\newcommand{\stitle}[1]{{\large{\textcolor{purple}{\emph{#1}}}}}
\DeclareMathAlphabet{\mathpgoth}{OT1}{pgoth}{m}{n}	
\newcommand{\cq}{\mathpgoth{cq} }
\newcommand{\cqstar}{\ensuremath{\mathpgoth{cq^{\mathbf{*}} }}\xspace}
\newcommand{\flookup}{\ensuremath{\mathsf{\mathpgoth{Flookup}}}\xspace}
\newcommand{\baloo}{\ensuremath{\mathrm{ba}\mathit{loo}}\xspace}
% \newcommand{\caulkp}{\ensuremath{\mathsf{\mathrel{Caulk}\mathrel{\scriptstyle{+}}}}\xspace}
\newcommand{\caulk}{\ensuremath{\mathsf{Caulk}}\xspace}
\newcommand{\plookup}{\ensuremath{\mathpgoth{plookup}}\xspace}
\newcommand{\srs}{\ensuremath{\mathsf{srs}}}
\newcommand{\tablegroup}{\ensuremath{\mathbb{H}}\xspace}
\newcommand{\V}{\ensuremath{\mathbf{V} }\xspace}
\newcommand{\zfin}{\ensuremath{z_{\mathrm{final}}}}
\newcommand{\rel}{\ensuremath{\mathcal{R}}}
\newcommand{\vk}{\ensuremath{\mathrm{vk} }}
\newcommand{\repr}{\ensuremath{\mathrm{repr} }}
\newcommand{\numreads}{\ensuremath{\mathbf{numreads} }}
\newcommand{\add}{\ensuremath{\mathbf{add} }}
\newcommand{\adds}{\ensuremath{\mathbf{adds} }}
\newcommand{\cnt}{\ensuremath{\mathpgoth{t} }}
\newcommand{\addcount}{\ensuremath{\mathpgoth{at} }}
\renewcommand{\read}{\ensuremath{\mathbf{read} }}
\newcommand{\reads}{\ensuremath{\mathbf{reads} }}
\renewcommand{\note}{\ensuremath{\mathfrak{n} }}
\newcommand{\vknext}{\ensuremath{\mathrm{vk_{next}} }}
\newcommand{\vkcur}{\ensuremath{\mathrm{vk_{cur}} }}
\newcommand{\args}{\ensuremath{\mathrm{args} }}
\newcommand{\stack}{\ensuremath{\mathsf{stack} }}
\newcommand{\argscur}{\ensuremath{\mathrm{args_{cur}} }}
\newcommand{\argsnext}{\ensuremath{\mathrm{args_{next}} }}
% \newcommand{\caulk}{{\mathsf{Caulk}}}
% \newcommand{\caulkp}{{\mathsf{\mathrel{Caulk}\mathrel{\scriptstyle{+}}}}}
\newcommand{\bigspace}{\ensuremath{\mathbb{V}}}
\newcommand{\mle}[1]{\ensuremath{\hat{#1}}}
\newcommand{\eq}{\ensuremath{\mathbf{eq}}}
\newcommand{\sumi}[1]{\sum_{i< #1}}
\newcommand{\com}{\ensuremath{\mathsf{com}}}
 \renewcommand{\polylog}{\ensuremath{\mathrm{polylog}}}
\newcommand{\X}{\ensuremath{\mathbf{X}}}
\newcommand{\basefold}{\ensuremath{\color{blue}\mathsf{BaseFold}}}
\newcommand{\inprod}[2]{\ensuremath{\left\langle #1,#2 \right \rangle}}
\newcommand{\negl}{\ensuremath{\mathsf{negl}(\lambda)}}
\newcommand{\mlpcs}{ml-PCS}

%\setbeamersize{text margin left=3mm,text margin right=3mm} 
\title{\large{Revisiting the IPA-sumcheck connection}}    % Enter your title between curly braces
\author{\small{Ariel Gabizon (based on work with Liam Eagen)}\\                 % Enter your name between curly braces
\tt{\footnotesize{Aztec Labs}                                       } }      % Enter your institute name between curly braces
\date{}                    % Enter the date or \today between curly braces
%\usefonttheme{professionalfonts}
%\usefonttheme[onlymath]{serif}
\begin{document}
\boldmath
% Creates title page of slide show using above information
\begin{frame}
  \titlepage
\end{frame}
% \begin{frame}
% tl;dr - we reduce IPA verifier time from linear to polylog using $\basefold$ on a polynomial with coeffs in the group.
%  \end{frame}
\begin{frame}
 \frametitle{Outline}   % Insert frame title between curly braces
 \begin{itemize}
  \item A few slides of motivation and context.
  \item Definitions reg multilinear polynomials.
  \item PCS landscape and where this work fits in.
  \item Construction.\nlnp 
 \end{itemize}

\end{frame}
\begin{frame}
 \frametitle{Succinct arguments in a nutshell}   % Insert frame title between curly braces
 Public program $T$, public output $z$.\\ \pause
 \vspace{0.4in}
 Want to prove ``I know input $x$ for program $T$ that generates output $z$.\\ \pause
 \vspace{0.4in}
Want proof size and verification time to be much smaller than run time of $T$. \\ 
{\small (SNARK:=Succinct Non-Interactive Argument of Knowledge)}\\ \pause
 \vspace{0.4in}
 Arithmeitization {\small [LFKN,......]}: Reduce claim to claim of form ''I know polynomials that satisfy some identity`` \pause
\end{frame}
\begin{frame}
 \frametitle{Succinct arguments in a nutshell}   % Insert frame title between curly braces
 Advantage of claims about polynomials is that suffice to check at one random point \\ \pause
 \vspace{0.4in}
 But need to solve ''chicken and egg problem``: Prover must commit to polynomials before knowing the challenge point. 
 \vspace{0.4in}
 
\end{frame}

\begin{frame}
 \frametitle{Multilinears polynomials}
Let $\X=(X_1,\ldots,X_k)$. By a ($k$-variate) multilinear polynomial $f(\X)$ over $\F$ we mean of \emph{individual} degree at most one.
e.g. 
\[\color{purple}f(\X) = X_1X_2 + X_3-5.\]\pause
\end{frame}
\begin{frame}

We define the well-known $\eq$ multilinear polynomial in $2k$ variables. 
\[\color{purple} \eq(x,y) \defeq \prod_{i=1}^{k} ( x_i y_i + (1-x_i)(1-y_i))\]\pause
We have for $x,y \in \bin{k}$, $\eq(x,y)=1$ when $x=y$ and $\eq(x,y)=0$ otherwise.\nl

For $z\in \bin{k}$, the functions $L_z(\X)=\eq(z,\X)$ are a ``Lagrange basis'' for multilinears.
\end{frame}
\begin{frame}
\frametitle{vectors and multilinears}
$n=2^k$.\nlnp
$f=(f_0,\ldots,f_{n-1})\in \F^n,z\in \F^k$
\[\color{purple}\mle{f}(z)\defeq \sum_{i<n} \eq(i,z) f_i\]\pause
 \emph{we identify $i$ with its binary representation.}
 \end{frame}
\begin{frame}
 \frametitle{Polynomial commitment schemes}
 A PCS is consists of a \nl
 \begin{itemize}
\item Procedure \textbf{commit:}  $f\in \F^n \to \com(f)$. \pause
\item Protocol \textbf{open}$(z\in \F^k,v\in \F,\cm)$:
$\prv$ convinces $\ver$ that it knows $f$ with $\com(f)=\cm$ and $\mle{f}(z)=v$.
\end{itemize}
\end{frame}
\begin{frame}
 \frametitle{PCS guarantees}
 \textbf{binding:} Efficient $\adv$ can't find $f_1\neq f_2$ with $\com(f_1)=\com(f_2)$.\nl
 \textbf{Knowledge soundness:} If efficient $\adv$ makes $\ver$ accept in $\open(\cm,z,v)$ it ``knows''  $f$ with $\com(f)=\cm$ and $\mle{f}(z)=v$.
%  

\end{frame}
\begin{frame}
 \frametitle{Discrete log hard groups}
 Additive group $\G$ with $|G|=|F|=r$ for prime $r$.\nl
 Uniformly chosen $G_0,\ldots,G_{n-1}\in \G$. 
For any efficient $\adv$ the prob. of finding non-zero $f\in \F^n$ with
 \[\color{purple}\sum f_i G_i=0\]
 is \negl.
\end{frame}
\begin{frame}
 \frametitle{The multilinear-PCS landscape }
\begin{table}[!htbp]\label{table:comparison}
% \begin{adjustbox}{width=1\textwidth}
	\begin{tabular}{l|}
	{\color{black}\textbf{Scheme type}}     \\ \hline
		\color{black}hash-based {\small [\basefold,WHIR,..]}
		         \\ \hline
		\color{black}DL-based(IPA){\small [Bootle et. al,bulletproofs,..]} \\ \hline
		
		\color{black} pairing-based {\small[Mercury,Samaritan,Dory,KZH-fold,..]} \\  \hline
	\end{tabular}
\end{table} 
 
 
\end{frame}
\begin{frame}
 \frametitle{The PCS landscape }
\begin{table}[!htbp]\label{table:comparison}
% \begin{adjustbox}{width=1\textwidth}
	\begin{tabular}{l|l|}
	{\color{black}\textbf{Scheme type}} & {\color{black}\textbf{Proof size}}    \\ \hline
		\color{black}hash-based 
		        & $\color{red}\polylog(n)$ \\ \hline
		\color{black}DL-based(IPA) & $\color{yellow} \log n$\\ \hline
		
		\color{black} pairing-based  & $\color{green}O(1)$\\  \hline
	\end{tabular}
\end{table} 
 
 
\end{frame}
\begin{frame}
 \frametitle{The PCS landscape }
\begin{table}[!htbp]\label{table:comparison}
% \begin{adjustbox}{width=1\textwidth}
	\begin{tabular}{l|l|}
	{\color{black}\textbf{Scheme type}} & {\color{black}\textbf{Commitment work}}    \\ \hline
		\color{black}hash-based 
		        & \color{green} Encode with ECC over small  field \\ \hline
		\color{black}DL-based & \color{yellow} MSM in DL-hard group \\ \hline
		
		\color{black} pairing-based  & \color{red} MSM in Pairing-friendly group\\  \hline
	\end{tabular}
\end{table} 
 
 
\end{frame}
\begin{frame}
 \frametitle{The PCS landscape }
\begin{table}[!htbp]\label{table:comparison}
% \begin{adjustbox}{width=1\textwidth}
	\begin{tabular}{l|l|}
	{\color{black}\textbf{Scheme type}} & {\color{black}\textbf{verification cost}}    \\ \hline
		\color{black}hash-based 
		        & \color{black} $\color{black} \polylog$ field work \\ \hline
		\color{black}DL-based & \color{red}$\color{red}O(n)$-MSM!!\\ \hline
		
		\color{black} pairing-based  & \color{black}  $\color{black}O(1)$ pairings and scalar mult\\  \hline
	\end{tabular}
\end{table} 
\end{frame}
\begin{frame}
 \frametitle{The PCS landscape }
\emph{accumulation:= reducing checking several evaluation claims into one.}\nlnp
\begin{table}[!htbp]\label{table:comparison}
% \begin{adjustbox}{width=1\textwidth}
	\begin{tabular}{l|l|}
	{\color{black}\textbf{Scheme type}} & {\color{black}\textbf{accumulation cost}}    \\ \hline
		\color{black}hash-based
		        &$\color{red} O(\polylog n)$ \\ \hline
		\color{black}DL-based & $\color{yellow} O(\log n)$\\ \hline
		
		\color{black} pairing-based  & $\color{green}  O(1)$ \\  \hline
	\end{tabular}
\end{table} 
\end{frame}
\begin{frame}
 \frametitle{This work:}
\begin{table}[!htbp]\label{table:comparison}
% \begin{adjustbox}{width=1\textwidth}
	\begin{tabular}{l|l|}
	{\color{black}\textbf{Scheme type}} & {\color{black}\textbf{verification cost}}    \\ \hline
		\color{black}hash-based 
		        & \color{black} $\color{black} \polylog$ \\ \hline
		\color{black}DL-based & \sout{\color{red}$\color{red}O(n)$-MSM!!}$\color{black}\to \polylog$ \\ \hline
		
		\color{black} pairing-based  & \color{black}  $\color{black}O(1)$ pairings and scalar mult\\  \hline
	\end{tabular}
\end{table} 
\end{frame}


\begin{frame}
 \frametitle{The sumcheck protocol \small\color{green}[LFKN]}
 $k$-variate poly $A(\X)$ with ind. degree $d$, and ``target value'' $C$.
 $\prv$ wants to prove
 \[\color{purple}\sum_{b\in \bin{k}}A(b) = C.\]\pause
 Sumcheck reduces this with $O(dk)$ communication to checking $A(r)=V$ for random $r\in \F^k$ and some $V$.\nl
 
 
 \emph{{\small\color{green} [BCS21]} showed sumcheck makes sense for polys over groups/rings/modules...}                                                
\end{frame}

\begin{frame}
 \frametitle{Polynomials over $\G$}
 \begin{figure}
  \includegraphics[width=260pt]{gpolys.png}
\end{figure}
 
\end{frame}
\begin{frame}
 \frametitle{MLPCS based on DL hardness as sumcheck {\small \color{green}[Bulletproofs, BCS21]}:}
\textbf{Setup:}
Choose random non-zero $G=(G_0,\ldots,G_{n-1})\in \G^n,P\in \G$.\nl
\textbf{Commitment:} $f\in \F^n$, $\com(f)=\sum_{i<n}f_i G_i$.\nl
\textbf{Openings:} next slide.
 \end{frame}
\begin{frame}
Given $\cm\in \G,z\in \F^k,v\in \F$ want to prove $\com(f)=\cm$ and $\mle{f}(z)=v$.\nl
Define the polynomial 
\[\color{purple}A(\X)\defeq    \mle{f}(\X) \mle{G}(\X) +  \eq(\X,z) \mle{f}(\X) P\]\pause
 When claim holds:
\[\color{purple}\sum_{b\in \bin{k}}\mle{f}(b) \mle{G}(b)+\eq(b,z)\mle{f}(b)P\]
\[\color{purple}= \sumi{n} f_i G_i + \eq(i,z) f_i P = \cm + \mle{f}(z) P\]
\end{frame}
\begin{frame}
$\prv$ and $\ver$ will run sumcheck on $A$ with target value $\cm+vP$.\nl
At end of sumcheck, $\ver$ needs to evaluate $A(r)=\mle{f}(r)\mle{G}(r) + \eq(r,z)\mle{f}(r)P$ for some $r\in \F^k$.\nl
 $\ver$ computes $\eq(r,z),\mle{G}(r)$. $\prv$ \emph{simply sends} $a=\mle{f}(r)$.\nl
\textit{The strange (and useful) thing:} When $\G$ has hard discrete-log this is sound!\nl
\emph{Drawback:} Computing $\mle{G}(r)$ is $n$-size MSM for $\ver$!
 \end{frame}
 \begin{frame}
  \frametitle{Mitigation from Halo: defer MSM}
  Given claims $\mle{G}(r_1)=V_1,\mle{G}(r_2)=V_2$.
  Can reduce them into one using sumcheck:\nl
  Recall \[\color{purple}\mle{G}(r) = \sum_{b\in \bin{k}} \eq(b,r) \mle{G}(b)\]\nl
  
  $\ver$  chooses random $\gamma$.
  Let $A(X)\defeq (\eq(X,r_1)+\gamma\cdot \eq(X,r_2)) \mle{G}(X)$.\nl
 
 We have 
   \[\color{purple} \sum_{b\in \bin{k}} A(X) =  \mle{G}(r_1)+\gamma \mle{G}(r_2)  \]\nl
   We run sumcheck on $A$  with target value $V_1+\gamma V_2$.

  \end{frame}
 \begin{frame}
  Reducing $\mle{G}(r_1)\stackrel{?}{=}V_1,\mle{G}(r_2)\stackrel{?}{=}V_2$ to $\mle{G}(r)\stackrel{?}{=}V$:
 
 \begin{enumerate}
  \item $\ver$  chooses random $\gamma$.
  \item Let $A(X)\defeq (\eq(X,r_1)+\gamma\cdot \eq(X,r_2)) \mle{G}(X)$.
  \item 
   $\prv$ and $\ver$ run sumcheck on $A$  with target value $V_1+\gamma V_2$.\pause
   \item Claim is reduced to $A(r)=V$, for some   $r\in \F,V'\in \G$.\pause
   \item $\ver$ computes $\eq(r,r_1),\eq(r,r_2)$ reducing the claim to $\mle{G}(r)=V$ for some $V$.
  
 
% \item After sumcheck $\ver$ is left with computing $A(r)$ requiring a \emph{single} evaluation $\mle{G}(r)$.
 \end{enumerate}
  \end{frame}

 \begin{frame}
  \frametitle{$\prv$ proving correctness of $\mle{G}(r)$}
\textit{Observation: If we have mlPCS for field-valued multilinears, where all ops on $f$'s vals are $\F$-linear, can also use on \emph{group valued} multilinear $G$. - e.g. \basefold}

 \end{frame}
 \begin{frame}
  \frametitle{FRI/ \basefold}
\textbf{Idea:} Think of $(G_0,\ldots,G_{n-1})$ as \emph{univariate}
\[\color{purple}g(X)\defeq \sum_{i=0}^{n-1} G_i X^i\]\pause
Send a merkle commitment to the values of $g$ on a $2n$-order subgroup.\nl
Recall we have 
\[\color{purple}g(X)=g_{even}(X^2)+X g_{odd}(X^2)\]\pause
Let $r=(r_1,\ldots,r_k)$. ``Fold'' $g$ by $r_1$: 
\[\color{purple}g_1(X)=(1-r_1)g_{even}(X)+ r_1 g_{odd}(X)\]
\end{frame}
\begin{frame}
Let $r=(r_1,\ldots,r_k)$. ``Fold'' $g$ by $r_1$: 
\[\color{purple}g_1(X)\defeq (1-r_1)g_{even}(X)+ r_1 g_{odd}(X)\]\pause
$\prv$ sends Merkle commitment to values of $g_1$ on $n$-order subgroup. \\ \noindent
$\ver$ checks on random locations $g_1$ is the correct folding.\nl

Now we have
\[\color{purple}\mle{G}(r)=\mle{g}_1(r_2,\ldots,r_k)\]\pause
% \textbf{Correlated Agreement Thm:}This is sound for random $r$.\pause
% 
% $\basefold$: Interleave with sumcheck to work for \emph{all} $r$

 \end{frame}
 
\begin{frame}
Now we have
\[\color{purple}\mle{G}(r)=\mle{g}_1(r_2,\ldots,r_k)\]\pause
\textbf{Correlated Agreement Thm:}This is sound for random $r$.\pause

$\basefold$: Interleave with sumcheck to work for \emph{all} $r$\pause

\textit{Future work:} Improve using WHIR

 \end{frame}
% \begin{frame}
%  ``Fold'' $g$ by $r_1$: 
% \[\color{purple}g_1(X)\defeq (1-r_1)g_{even}(X)+ r_1 g_{odd}(X)\]
% $\prv$ sends Merkle commitment to values of $g_1$ on $n$-order subgroup. \\ \noindent
% $\ver$ checks on random locations $g_1$ is the correct folding.\nl
% 
% Now we have
% \[\color{purple}\mle{G}(r)=\mle{g}_1(r_2,\ldots,r_k)\]\pause
% \textbf{Correlated Agreement Thm:}This is sound for random $r$.\\ \noindent
% $\basefold$: Interleave with sumcheck to work for \emph{all} $r$
% 
% 
%  \end{frame}
 \begin{frame}
 \begin{figure}
  \includegraphics[width=260pt]{ipatitle.png}
\end{figure}
 See paper on eprint for details. Thanks!
  
 \end{frame}

\end{document}
