\documentclass[shadesubsections,trans,14pt,mathserif]{beamer}
\usepackage[danish]{babel}	
%\usepackage[T1]{fontenc}
%\usepackage{fourier}
% Dokumentets sprog
%\usepackage{mathtools}
%\usepackage{pxfonts}
\usepackage{eulervm}
% Class options include: notes, notesonly, handout, trans,
%                        hidesubsections, shadesubsections,
%                        inrow, blue, red, grey, brown

% Theme for beamer presentation.
%\usepackage{beamertheme} 
% Other themes include: beamerthemebars, beamerthemelined, 
%                       beamerthemetree, beamerthemetreebars  
\newcommand{\adv}{\ensuremath{\mathcal A}}
\newcommand{\F}{\ensuremath{\mathbb F}}
\newcommand{\set}[1]{\ensuremath{\left\{#1\right\}}}


\title{\LARGE{On knowledge assumptions in recent zk-SNARK constructions}}    % Enter your title between curly braces
\author{\Large{Ariel Gabizon}}                 % Enter your name between curly braces
\institute{\normalsize{Protocol Labs}}      % Enter your institute name between curly braces
\date{}                    % Enter the date or \today between curly braces
%\usefonttheme{professionalfonts}
%\usefonttheme[onlymath]{serif}
\begin{document}
\boldmath
% Creates title page of slide show using above information
\begin{frame}
  \titlepage
\end{frame}
\note{Talk for 30 minutes} % Add notes to yourself that will be displayed when
                           % typeset with the notes or notesonly class options

%\section[Outline]{}

% Creates table of contents slide incorporating
% all \section and \subsection commands
%\begin{frame}
  %\tableofcontents
%\end{frame}

\begin{frame}
  \frametitle{Knowledge assumptions:}   % Insert frame title between curly braces
\textbf{Standard crypto assumption} - you can't do $X$\\
 \vspace{0.4in}
\textbf{Knowledge assumption} - if you did $X$, you must have did it in way $Y$.
 \vspace{0.4in}

\end{frame}



\begin{frame}
  \frametitle{Basic Knowledge of Exponent Assumption (Damg{\aa}rd 91)}   % Insert frame title between curly braces
 $\adv$ is given $(g,g^\alpha)$ for uniform $\alpha \in \F$.\\
 \vspace{0.4in}
Challenged to make a new pair of ``ratio $\alpha$'' $(g^c,g^{\alpha c})$.

 \vspace{0.4in}

\textbf{KEA:} If $\adv$ succeeds then he ``knows'' $c$.
 
\end{frame}

\begin{frame}
 Suppose $\adv$ is given $(g,g^x,g^\alpha,g^{\alpha \cdot x})$.\\
 \vspace{0.4in}
Can output $(g^c,g^{\alpha\cdot c})$ for $c=c_1 + c_2\cdot x$. 
\end{frame}


\begin{frame}
  \frametitle{$d$-power KEA (Groth,10)}   % Insert frame title between curly braces
 Given $(g,g^x,\ldots, g^{x^d},g^\alpha,g^{\alpha\cdot x},\ldots,g^{\alpha\cdot x^d})$ for uniform $\alpha,x \in \F$.\\
 \vspace{0.3in}
If $\adv$ produces $(g^c,g^{\alpha c})$,
then he ``knows'' polynomial $A$ of degree $\leq d$ with
$c=A(x)$.
 
\end{frame}
\begin{frame}
  \frametitle{$d$-power KEA (Groth,10)}   % Insert frame title between curly braces
 Given $(g,g^x,\ldots, g^{x^d},g^\alpha,g^{\alpha\cdot x},\ldots,g^{\alpha\cdot x^d})$ for uniform $\alpha,x \in \F$.\\
 \vspace{0.3in}
If $\adv$ produces $(g^c,g^{\alpha c})$,
then he ``knows'' polynomial $A$ of degree $\leq d$ with
$c=A(x)$.\\
 \vspace{0.4in}
 \emph{Enables ``blind verifiable polynomial evaluation'' in SNARKs}
\end{frame}


\begin{frame}
 \frametitle{Abstracting and Generalizing KEA}
 \begin{itemize}
  \item $[x]:=g^x$ -  \textbf{encoding} of $x$.
  \vspace{0.3in}
 \end{itemize}
\end{frame}

\begin{frame}
 \frametitle{Abstracting and Generalizing KEA}
 \begin{itemize}
  \item $[x]:=g^x$ -  \textbf{encoding} of $x$.
  \vspace{0.3in}
  \item Challenge Equation: $Y_2=\alpha\cdot Y_1$.
   \vspace{0.3in}
 \end{itemize}
\end{frame}
\begin{frame}
 \frametitle{Abstracting and Generalizing KEA}
 \begin{itemize}
  \item $[x]:=g^x$ -  \textbf{encoding} of $x$.
  \vspace{0.3in}
  \item Challenge Equation: $Y_2=\alpha\cdot Y_1$.
   \vspace{0.3in}
\item Base set: $\set{1,x,\ldots,x^d,\alpha,\alpha\cdot x,\ldots,\alpha\cdot x^d}$
 \end{itemize}
\end{frame}

\begin{frame}
 \frametitle{Abstracting and Generalizing KEA}
 \begin{itemize}
  \item $[x]:=g^x$ -  \textbf{encoding} of $x$.
  \vspace{0.2in}
  \item Challenge Equation: $Y_2=\alpha\cdot Y_1$.
   \vspace{0.2in}
\item Base set: $\set{1,x,\ldots,x^d,\alpha,\alpha\cdot x,\ldots,\alpha\cdot x^d}$
 \end{itemize}
\vspace{0.3in}
 \textbf{Generic Assumption:} \emph{Given encoded challenge set, if $\adv$ outputs encodings of $c,c'$ satisfying challenge equation;\\
 Then he knows to write $c,c'$ as linear combination of base set s.t. equation holds as pol. identity in $\alpha$.}
 \end{frame}


 \begin{frame}
 \frametitle{Keep the equation, change the challenge set}
 
 \[\alpha_1 \cdot Y_1  = Y_{2}\]
  \vspace{0.3in}
\[\set{u_i(x),\alpha\cdot u_i(x)},\]
for specific polys $u_i(X)$\\
 
 
 
 \end{frame}

 
 
 \begin{frame}
 \frametitle{Keep the equation, change the challenge set}
 
 \[\alpha_1 \cdot Y_1  = Y_{2}\]
  \vspace{0.3in}
  
  Base set:
\[\set{u_i(x),\alpha\cdot u_i(x)},\]
for specific polys $u_i(X)$\\
  \vspace{0.3in}
  \emph{Intuition: allows to check $c$ is combination of these specific polynomials evaluted at $x$}
  
  
  
  
  %  
%  \begin{lemma}
%   Doesn't lead to stronger assumption than two-variable version
%  \end{lemma}

 
 
 \end{frame}
 
%  \begin{frame}
%  \frametitle{Mutli-variate challenge equations}
%  
%  \[\alpha_1 \cdot Y_1 + \ldots + \alpha_t \cdot Y_t = Y_{t+1}\]
%   \vspace{0.3in}
%  
%  \begin{lemma}
%   Doesn't lead to stronger assumption than two-variable version
%  \end{lemma}
% 
%  
%  \emph{Intuition: allows to batch-verify many blind evaluations}
%  
%  \end{frame}
% 
 
 
 
 \begin{frame}
 \frametitle{Quadratic equation assumptions [Implicit in Groth, 2016]}
 
 
 \[Y_1\cdot Y_2= \alpha\beta + \delta \cdot Y_{3}\]
  \vspace{0.3in}
 
%[Groth,2016] zk-SNARK in GG model (implicitly) based on  assumption from this equation.
 
 Base set: $\set{\beta\cdot u_i(X),\alpha\cdot v_i(X), w_i(X),
 \frac{\beta \cdot u_i+\alpha\cdot v_i+w_i}{\delta}}$
  \vspace{0.3in}
 
 
 \end{frame}

 
 
 
 \begin{frame}
 \frametitle{Quadratic equation assumptions [Implicit in Groth, 2016]}
 
 
 \[Y_1\cdot Y_2= \alpha\beta + \delta \cdot Y_{3}\]
  \vspace{0.3in}
 
%[Groth,2016] zk-SNARK in GG model (implicitly) based on  assumption from this equation.
 
 Base set: $\set{\beta\cdot u_i(X),\alpha\cdot v_i(X), w_i(X),
 \frac{\beta \cdot u_i+\alpha\cdot v_i+w_i}{\delta}}$
  \vspace{0.3in}
 
 \emph{Intuition: allows to check at once three proof elements are the same combination of three sets of polynomials }
 
 \end{frame}
\begin{frame}
 \textbf{Generic Group Model:}
 $\adv$ can only generate new elements via natural group operations.\\
 
\vspace{0.3in}
% \textbf{Algebraic Group Model [Fuchsbauer, Kiltz, Loss]:}
% $\adv$ can generate arbitrary new elements, but must then 
 
 \begin{definition}
  A ``low-degree world'' is one where all information seen by $\adv$ is encodings of uniform inputs evaluated at low-degree polynomials. 
 \end{definition}
 

 \end{frame}
\begin{frame}
 \vspace{0.3in}
 \begin{lemma}[implicit - Groth, 2016]
  In a low-degree world, generic assumption holds for any polynomial-degree challenge equation and challenge set.
 \end{lemma}

 

 \end{frame}

 \begin{frame}
 \frametitle{``Asymmetric'' assumptions [Groth-Maller, 2017]:}
Use trivial equation
\[Y_1=Y_2,\]
but require encoding in distinct groups with no homomorphism.

 

 \end{frame}

  \begin{frame}
 \frametitle{``Asymmetric'' assumptions [Groth-Maller, 2017]:}


\textbf{[GM]:}Using circuits with squaring instead of mult. gives  simulation-extractable zk-SNARKs as small as [Groth,2016] - 3 group elements.
 \vspace{0.3in}

\emph{However, prover computations significantly larger than [Groth,2016] cause of move to squarings}
 

 \end{frame}


 \begin{frame}
  \textbf{New result:}
  Assuming only $d-PKE$.
  zk-SNARK with $4$-group element proofs; prover run time close to [Groth,2016]\\
  
  Same num. of $G_2$ operations. $n$ more $G_1$ operations.\\
  
  $n$= num. of mult gates.
  

  
 \end{frame}

 
 
 \begin{frame}
 \frametitle{Result based on multivariate challenge equation:}
 
 \[\alpha_1 \cdot Y_1 + \ldots + \alpha_t \cdot Y_t = Y_{t+1}\]
  \vspace{0.3in}

 
   \emph{Intuition: can allow to verify many blind evaluations with one extra element}

 

 
\end{frame}  
 
 
 
 
%  \begin{frame}
%  \frametitle{Result based on:}
%  \begin{lemma}
%   Assumption from multivariate 
%  \[\alpha_1 \cdot Y_1 + \ldots + \alpha_t \cdot Y_t = Y_{t+1}\]
%  not stronger than bi-variate version.
%  \end{lemma}
% 
%  
% \end{frame}  
%  
%  
 \begin{frame}
 
 
 
 
 
 \begin{lemma}
  Assumption from multivariate 
 \[\alpha_1 \cdot Y_1 + \ldots + \alpha_t \cdot Y_t = Y_{t+1}\]
 not stronger than bi-variate version.
 \end{lemma}

 
\end{frame}  
  
  
  
  %  
%  \begin{lemma}
%   Doesn't lead to stronger assumption than two-variable version
%  \end{lemma}

 
 

 
 
 
% 
% \subsection{Simple slide with three points shown in succession}
% 
% \begin{frame}
%   \frametitle{Simple slide with three points shown in succession}   % Insert frame title between curly braces
% 
%   \begin{itemize}
%   \item<1-> Point 1 (Click ``Next Page'' to see Point 2) % Use Next Page to go to Point 2
%   \item<2-> Point 2  % Use Next Page to go to Point 3
%   \item<3-> Point 3
%   \end{itemize}
% \end{frame}
% \note{Speak clearly}  % Add notes to yourself that will be displayed when
%                       % typeset with the notes or notesonly class options
% 
% 
% \section{Slide with two columns: items and a graphic}
% 
% \begin{frame}
%   \frametitle{Slide with two columns: items and a graphic}   % Insert frame title between curly braces
%   \begin{columns}[c]
%   \column{2in}  % slides are 3in high by 5in wide
%   \begin{itemize}
%   \item<1-> First item
%   \item<2-> Second item
%   \item<3-> ...
%   \end{itemize}
%   \column{2in}
%   \framebox{Insert graphic here % e.g. \includegraphics[height=2.65in]{graphic}
%   }
%   \end{columns}
% \end{frame}
% \note{The end}       % Add notes to yourself that will be displayed when
% 		     % typeset with the notes or notesonly class options

\end{document}
