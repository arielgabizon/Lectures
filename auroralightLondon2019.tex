\documentclass[shadesubsections,trans,14pt,mathserif]{beamer}
\usepackage[danish]{babel}	
%\usepackage[T1]{fontenc}
%\usepackage{fourier}
% Dokumentets sprog
%\usepackage{mathtools}
%\usepackage{pxfonts}
\usepackage{eulervm}
% Class options include: notes, notesonly, handout, trans,
%                        hidesubsections, shadesubsections,
%                        inrow, blue, red, grey, brown

% Theme for beamer presentation.
%\usepackage{beamertheme} 
% Other themes include: beamerthemebars, beamerthemelined, 
%                       beamerthemetree, beamerthemetreebars  
\newcommand{\adv}{\ensuremath{\mathcal A}}
\newcommand{\F}{\ensuremath{\mathbb F}}
\newcommand{\set}[1]{\ensuremath{\left\{#1\right\}}}
\newcommand{\sett}[2]{\ensuremath{\left\{#1\right\}_{#2}}}
\newcommand{\enc}[1]{\ensuremath{\left[#1\right ]}}
\newcommand{\cm}{\ensuremath{\mathsf{cm}}}
\newcommand{\open}[1]{\ensuremath{\mathsf{open}(#1)}}
\newcommand{\verify}[1]{\ensuremath{\mathsf{verify}(#1)}}
\newcommand{\defeq}{\ensuremath{:=}}
\newcommand{\helper}{\ensuremath{\mathcal{H}}}
\newcommand{\ver}{\ensuremath{\mathcal{V}}}
\newcommand{\prv}{\ensuremath{\mathcal{P}}}

\title{\LARGE{Components of recent universal SNARKs}}    % Enter your title between curly braces
\author{\Large{Ariel Gabizon}}                 % Enter your name between curly braces
\institute{\normalsize{Protocol Labs}}      % Enter your institute name between curly braces
\date{}                    % Enter the date or \today between curly braces
%\usefonttheme{professionalfonts}
%\usefonttheme[onlymath]{serif}
\begin{document}
\boldmath
% Creates title page of slide show using above information
\begin{frame}
  \titlepage
\end{frame}
\note{Talk for 30 minutes} % Add notes to yourself that will be displayed when
                           % typeset with the notes or notesonly class options

%\section[Outline]{}

% Creates table of contents slide incorporating
% all \section and \subsection commands
%\begin{frame}
  %\tableofcontents
%\end{frame}


\begin{frame}
  \frametitle{30 seconds of philosophy: \\ SNARKs and the meaning of life}   % Insert frame title between curly braces

  
  As the world gets increasingly distributed and digital, SNARKs help us ``keep a grip on reality'', by ensuring the data we receive is anchored in reality.\\
 \vspace{0.4in}
 In this context, they are vast generalization of hashes and digital signatures.
 \vspace{0.4in}

\end{frame}



\begin{frame}
\frametitle{Evaluating prover polynomials succinctly}   % Insert frame title between curly braces
Since the beginning of time (LFKN, 1990) humanity has been trying to verify prover polynomial evaluations.\\
\end{frame}


\begin{frame}
\frametitle{Evaluating prover polynomials succinctly}   % Insert frame title between curly braces
%Since the beginning of time (LFKN, 1989) humanity has been trying to verify prover polynomial evaluations.\\
 \vspace{0.4in}
\textbf{Verifer:} - ``Choose a degree 10 polynomial $f$ and keep it in your head''\\
\textbf{Prover:} - ``OK''\\
 \textbf{Verifer:} - ``What is $f(7)$?''\\
\textbf{Prover:} - ``$10$''\\
\end{frame}


\begin{frame}
  \frametitle{Evaluating prover polynomials succinctly}   % Insert frame title between curly braces
 
\textbf{The hard-working honest way} - Low degree testing [BFL91,...] \small{(also PCPPs/IOPPs)}
%(\footnotesize{(and PCPPs,IOPPS)}) 

\end{frame}




\begin{frame}
  \frametitle{Evaluating prover polynomials succinctly}   % Insert frame title between curly braces
%Since the beginning of time (1989) humanity has been trying to verify prover polynomial evaluations.
 %\vspace{0.4in}
 
\textbf{The hard-working honest way} - Low degree testing/PCPP/IOPPs

 \vspace{0.4in}
\textbf{The correct* way} - Polynomial commitment scheme (with SRS).


\vspace{0.8in}
\footnotesize{*The opinions expressed in this presentation are objective reality}
\end{frame}



\begin{frame}
\frametitle{Notation:``an encoding of $x$''}   % Insert frame title between curly braces
%Since the beginning of time (1989) humanity has been trying to verify prover polynomial evaluations.
 %\vspace{0.4in}
 
%\textbf{Notation:}
$\enc{x}\defeq x\cdot g = g+\ldots+g$ ($x$ times).
 \vspace{0.4in}
 
\end{frame}

\begin{frame}
\frametitle{Notation:``an encoding of $x$''}   % Insert frame title between curly braces
%Since the beginning of time (1989) humanity has been trying to verify prover polynomial evaluations.
 %\vspace{0.4in}
 
%\textbf{Notation:}
$\enc{x}\defeq x\cdot g = g+\ldots+g$ ($x$ times).\\
 \vspace{1in}
\emph{\small{You can imagine it means $g^x$, if it helps you sleep better.}} 
\end{frame}



\begin{frame}
  \frametitle{The KZG polynomial commitment scheme}   % Insert frame title between curly braces
 SRS: \enc{1},\enc{x},\ldots,\enc{x^d}, for random $x\in \F$.\\
 \vspace{0.4in}
 $f(X) = \sum_{i=0}^d a_i X^i$\\
 \vspace{0.4in}
 $\cm(f)\defeq  \sum_{i=0}^d a_i \enc{x^i}=  \enc{f(x)}$\\
 \vspace{0.4in}
 
\end{frame}
\begin{frame}
 SRS: \enc{1},\enc{x},\ldots,\enc{x^d},\\
 for random $x\in \F$.
 \vspace{0.4in}
 
 $\cm(f)\defeq   \enc{f(x)}$\\
 \vspace{0.4in}
$\open{f,i}\defeq \enc{h(x)}$, where
$h(X)\defeq \frac{f(X)-f(i)}{X-i}$
 \vspace{0.4in}
 
\end{frame}


\begin{frame}
 $\cm(f)\defeq   \enc{f(x)}$\\
 \vspace{0.4in}
$\open{f,i}\defeq \enc{h(x)}$, where
 $h(X)\defeq \frac{f(X)-f(i)}{X-i}$\\
 \vspace{0.4in}
\textbf{exercise} - using pairing, define:\\
 \vspace{0.2in}
 $\verify{\cm,\pi,z,i}$,
 where $z$ is allegedly $f(i)$
 \end{frame}




\begin{frame}

 $\cm(f)\defeq   \enc{f(x)}$\\
 \vspace{0.4in}
$\open{f,i}\defeq \enc{h(x)}$, where
 $h(X)\defeq \frac{f(X)-f(i)}{X-i}$\\
 \vspace{0.4in}
 $\verify{\cm,\pi,z,i}:$
\[e(\cm-\enc{z},\enc{1}) \stackrel{?}{=} e(\pi, \enc{x-i})\]
\end{frame}


 


\begin{frame}

 $\cm(f)\defeq   \enc{f(x)}$\\
 \vspace{0.4in}
$\open{f,i}\defeq \enc{h(x)}$, where
 $h(X)\defeq \frac{f(X)-f(i)}{X-i}$\\
 \vspace{0.4in}
 $\verify{\cm,\pi,z,i}:$
\[e(\cm-\enc{z},\enc{1}) \stackrel{?}{=} e(\pi, \enc{x-i})\]
 \vspace{0.4in}
\textbf{Thm}{\footnotesize{[KZG,MBKM]}}: \emph{This works.}
 \end{frame}

\begin{frame}
  \frametitle{Application: Proof of retrievability}
\end{frame}

\begin{frame}
  \frametitle{Application: Proof of retrievability}
 \emph{Server:} Commit to file as coefficients \set{a_0,\ldots,a_d} of polynomial $f$.\\
 \vspace{0.4in}
 \emph{Client:} Time-to-time query $f$ at a random point $r$.\\
 \vspace{0.4in}
\end{frame}


\begin{frame}
  \frametitle{Application: Proof of retrievability}
 \emph{Server:} Commit to file as coefficients \set{a_0,\ldots,a_d} of polynomial $f$.\\
 \vspace{0.4in}
 \emph{Client:} Time-to-time query $f$ at a random point $r$.\\
 \vspace{0.4in}
 \emph{Probability of server computing $f(r)$ correctly w/o remembering all coeffs of $f$ is negligible.}
\end{frame}




\begin{frame}
  \frametitle{Application: The Sonic helper \small{[MBKM19]}}
 Bi-variate $S(X,Y)$, evaluation points \sett{(x_j,y_j)}{j\in \set{1..t}}, values \set{z_j} \\
 \end{frame}

\begin{frame}
  \frametitle{Application: The Sonic helper \small{[MBKM19]}}
 Bi-variate $S(X,Y)$, evaluation points \sett{(x_j,y_j)}{j\in \set{1..t}}, values \set{z_j} \\
 \vspace{0.4in}
 Helper $\helper$ wants to convince verifier $\ver$ that $\forall j\in \set{1..t}$, $S(x_j,y_j)=z_j$.\\
 \vspace{0.4in}
\end{frame}

\begin{frame}
  \frametitle{Application: The Sonic helper \small{[MBKM19]}}
 Bi-variate $S(X,Y)$, evaluation points \sett{(x_j,y_j)}{j\in \set{1..t}}, values \set{z_j} \\
 \vspace{0.4in}
 Helper $\helper$ wants to convince verifier $\ver$ that $\forall j\in \set{1..t}$, $S(x_j,y_j)=z_j$.\\
 \vspace{0.4in}
$\ver$'s work: only \emph{one} evaluation of $S$!
 \end{frame}

 
\begin{frame}
  \frametitle{The Sonic helper \small{[MBKM19]}}
 \begin{enumerate}
  \item $\forall j$, $\helper$ sends $S_j \defeq \cm(S(X,y_j))$.
 \end{enumerate}
 
 \end{frame}
\begin{frame}
  \frametitle{The Sonic helper \small{[MBKM19]}}
 \begin{enumerate}
  \item $\forall j$, $\helper$ sends $S_j \defeq \cm(S(X,y_j))$.
 \end{enumerate}
 \vspace{0.4in}
\emph{If $\helper$ would convince $\ver$ $S_j$'s are correct, he could just open them at $x_j$. } 
 \end{frame}

\begin{frame}
  \frametitle{The Sonic helper \small{[MBKM19]}}
 \begin{enumerate}
  \item $\forall j$, $\helper$ sends $S_j \defeq \cm(S(X,y_j))$.
\item $\ver$ chooses random $u\in \F$.
\item $\helper$ sends $C\defeq \cm(S(u,Y))$.

  \end{enumerate}
 \vspace{0.4in}
 \end{frame}

 
 \begin{frame}
  \frametitle{The Sonic helper \small{[MBKM19]}}
 \begin{enumerate}
  \item $\forall j$, $\helper$ sends $S_j \defeq \cm(S(X,y_j))$.
\item $\ver$ chooses random $u\in \F$.
\item $\helper$ sends $C\defeq \cm(S(u,Y))$.
\item $\forall j$, $\helper$ opens $C$ at $y_j$ and $S_j$ at $u$. $\ver$ checks they open to same value.
  \end{enumerate}
 \vspace{0.4in}
\emph{At this point $\ver$ knows $S_j$'s are correct \textbf{if} $C$ is correct.}
 \end{frame}

 \begin{frame}
  \frametitle{The Sonic helper \small{[MBKM19]}}
 \begin{enumerate}
  \item $\forall j$, $\helper$ sends $S_j \defeq \cm(S(X,y_j))$.
\item $\ver$ chooses random $u\in \F$.
\item $\helper$ sends $C\defeq \cm(S(u,Y))$.
\item $\forall j$, $\helper$ opens $C$ at $y_j$ and $S_j$ at $x_j$. $\ver$ checks they open to same value.
\item $\ver$ chooses random $v\in \F$ and computes $s(u,v)$.
\end{enumerate}
 \vspace{0.4in}
 \end{frame}

 
 
 \begin{frame}
\frametitle{The Sonic helper \small{[MBKM19]}}
 \begin{enumerate}
  \item $\forall j$, $\helper$ sends $S_j \defeq \cm(S(X,y_j))$.
\item $\ver$ chooses random $u\in \F$.
\item $\helper$ sends $C\defeq \cm(S(u,Y))$.
\item $\forall j$, $\helper$ opens $C$ at $y_j$ and $S_j$ at $x_j$. $\ver$ checks they open to same value.
\item $\ver$ chooses random $v\in \F$ and computes $s(u,v)$.
\item $\helper$ opens $C$ at $v$, and $\ver$ checks it opens to $s(u,v)$.
\end{enumerate}
 \vspace{0.4in}
 \end{frame}

 %\small{[BCRSV19, G19]}
  \begin{frame}
\frametitle{3rd application: sumchecks }
 \end{frame}
 
  \begin{frame}
\frametitle{3rd application: sumchecks}
$\prv$ has polynomial $f$. $H\subset \F$. \\
 \vspace{0.4in}
$\ver$ wants to check:
\[\sum_{x\in H} f(x) =0\]

\end{frame}


  \begin{frame}
\frametitle{The Aurora trick for sumcheck \small{[BCRSV19]}}
\begin{lemma}
 When $H\subset \F$ is a multiplicative subgroup of size $n$, and $deg(g)<n$
\[\sum_{x\in H} g(x) =0\]
iff $g$ has constant coefficient $0$.
\end{lemma}
 \vspace{0.4in}

\end{frame}


  \begin{frame}
\frametitle{The Aurora trick for sumcheck \small{[BCRSV19]}}
$Z_H \defeq \prod_{x\in H}(X-x)$
 \vspace{0.4in}

By lemma suffices to show $g_1,g_2$, $deg(g_2)<n-1$ such that
 \[ f(X)\equiv g_1(X)\cdot Z_H(X) +X\cdot g_2(X) \]
\end{frame}


  \begin{frame}
\frametitle{AuroraLight \small{[G19]}}
Check $f$ has this form by sending commitments to $g_1,g_2$,
and openning $f,g_1,g_2$ at random point.\\
 \vspace{0.4in}
% Using Aurora+Sonic ideas:
% \begin{corollary}
%  Universal SNARK with prover almost as fast as {\small{[Groth16]}}
%  (But longer proofs than Sonic, and no fully succinct mode).
% \end{corollary}
% 
\end{frame}

  \begin{frame}
\frametitle{AuroraLight \small{[G19]}}
Check $f$ has this form by sending commitments to $g_1,g_2$,
and openning $f,g_1,g_2$ at random point.\\
 \vspace{0.4in}
Using Aurora+Sonic ideas:
\begin{corollary}
 Universal SRS SNARK with prover almost as fast as {\small{[Groth16]}}
 (But longer proofs than Sonic, and no fully succinct mode).
\end{corollary}

\end{frame}



\end{document}
