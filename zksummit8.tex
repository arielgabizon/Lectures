\documentclass[shadesubsections,compress,14pt,mathserif]{beamer}
\usepackage[danish]{babel}	
\usepackage{tikz}
\usetikzlibrary{shapes, positioning}
\usenavigationsymbolstemplate{}
\usepackage{pgfplots}
\usepackage[absolute,overlay]{textpos}
\usepackage{amsthm}
%\usepackage[T1]{fontenc}
% \usepackage{fullpage}
% Dokumentets sprog
%\usepackage{mathtools}
%\usepackage{pxfonts}
\usepackage{eulervm}

% Class options include: notes, notesonly, handout, trans,
%                        hidesubsections, shadesubsections,
%                        inrow, blue, red, grey, brown

% Theme for beamer presentation.
%\usepackage{beamertheme} 
% Other themes include: beamerthemebars, beamerthemelined, 
%                       beamerthemetree, beamerthemetreebars  
\newcommand{\adv}{\ensuremath{\mathcal A}}
\newcommand{\F}{\ensuremath{{\mathbb F}_p}}
\newcommand{\Z}{\ensuremath{{\mathbb Z}}\xspace}
\newcommand{\Fclosure}{\ensuremath{{\overline{\mathbb{F}}}_p}}
\newcommand{\set}[1]{\ensuremath{\left\{#1\right\}}}
\newcommand{\sett}[2]{\ensuremath{\left\{#1\right\}_{#2}}}
\newcommand{\enc}[1]{\ensuremath{\left[#1\right ]}}
\newcommand{\cm}{\ensuremath{\mathsf{cm}}}
\newcommand{\open}[1]{\ensuremath{\mathsf{open}(#1)}}
\newcommand{\verify}[1]{\ensuremath{\mathsf{verify}(#1)}}
\newcommand{\defeq}{\ensuremath{:=}}
\newcommand{\helper}{\ensuremath{\mathcal{H}}}
\newcommand{\ver}{\ensuremath{\mathcal{V}}}
\newcommand{\prv}{\ensuremath{\mathcal{P}}}
 \newcommand{\polysofdeg}[1]{\F_{< #1}[X]}
%  \newcommand{\endoss}{\ensuremath{\mathrm{END}_E}}
 \newcommand{\hl}[1]{\textbf{\textit{#1}}}
 \newcommand{\polys}{\F[X]}
\newcommand{\acc}{{\mathbf{acc}}}
\newcommand{\ideal}{\mathbf{I}}
\newcommand{\gen}{\alpha}
\newcommand{\spac}{\\  \vspace{0.2in} \noindent}
\newcommand{\polylog}{\ensuremath{\mathsf{polylog}}\xspace}
% \renewcommand{\bf}{\begin{frame}}
% \newcommand{\ef}{\end{frame}}
%\setbeamersize{text margin left=3mm,text margin right=3mm}  
\newcommand{\nl}{\\ \pause \vspace{0.2in}}
\newcommand{\nlnp}{\\ \vspace{0.2in}}
\newcommand{\stitle}[1]{{\large{\textcolor{purple}{\emph{#1}}}}}
\title{\large{A gentle  introduction to Hasse's theorem}}    % Enter your title between curly braces
\author{{Ariel Gabizon}\\                 % Enter your name between curly braces
\tt{\small{Aztec} } }      % Enter your institute name between curly braces
\date{}                    % Enter the date or \today between curly braces
%\usefonttheme{professionalfonts}
%\usefonttheme[onlymath]{serif}
\begin{document}
\boldmath
% Creates title page of slide show using above information
\begin{frame}
  \titlepage
\end{frame}
                           % typeset with the notes or notesonly class options

%\section[Outline]{}

% Creates table of contents slide incorporating
% all \section and \subsection commands
%\begin{frame}
  %\tableofcontents
%\end{frame}

\begin{frame}
\frametitle{Motivation}
Opening the black-box of elliptic curves can 
bring interesting results.
 \begin{figure}
  \includegraphics[width=150pt]{zkecc.png}
% \end{figure}
%  \begin{figure}
  \includegraphics[width=150pt]{ecfft.png}
\end{figure}

\end{frame}
\begin{frame}
Curve $E$ over \F.
\[y^2=x^3+ax+b\]\pause
 $N\defeq$ number of points over \F.\pause
 \begin{itemize}
 \item $N\geq 1$ - cause always have point at infinity.\pause
 \item $N\leq 2p+1$ - cause at most two $y$'s for each $x\in \F$.\nl
\end{itemize}
\hl{Hasse's thm:} Set $a\defeq p+1-N$. Then $|a|\leq 2\sqrt p$.
\end{frame}
\begin{frame}
\frametitle{Hasse's thm is foundational for EC cryptography}
\hl{Hasse's thm:} $|p+1 -N|\leq 2\sqrt p$.\nlnp
\textit{Want $E$ of size $N\sim 2^m$?} \nl


Just pick prime $p$ with $m$ bits,
and take \emph{any} elliptic curve $E$ over \F.\nl
% any $a,b\in\F$  with $4a^3+27b^2\neq 0$\nlnp

% and take curve $y^2=x^3+ax+b$.\nl
\end{frame}

\begin{frame}
\frametitle{The proof begins with some junior-high algebra}
set $a\defeq p+1-N$.

and $L(X)
\defeq X^2-aX+p
$\nl


% Write $L(X)=(X-\omega_1)(X-\omega_2), \omega_i\in \mathbb{C}$. \pause
% Enough to show $\omega_1,\omega_2\notin {\mathbb R}$:\nl
Enough to show $L$'s roots are not in $\mathbb{R}$:\nl

It would imply:
\[\Delta=
a^2-4p<0
\rightarrow |a|<2\sqrt p\]\nl
% (for prime p cannot hit the = case cause 2sqrt(p) is not an integer)
% We begin by showing an \textbf{endomorphism} is a ``root'' of $L$.
\end{frame}
\begin{frame}
$a\defeq p+1-N$. $L(X)= X^2-aX+p$\nl
 \hl{Claim:} When $p$ is prime, $L$'s roots cannot be in $\mathbb{Z}$.\nl
 \hl{proof:}
 Write $L(X)=(X-\omega_1)(X-\omega_2)$.
%  where $\omega_1,\omega_2\in \mathbb{C}$.
 We have $p=\omega_1\cdot \omega_2$, $a=\omega_1+\omega_2$.\nl
 Only options for $\set{\omega_1,\omega_2}$:\nl
 \begin{itemize}
  \item $\set{1,p}$,  but then $a=p+1\Rightarrow N=0$.\pause
  \item $\set{-1,-p}$, but then $N=2p+2$. 
 \end{itemize}

\end{frame}


\begin{frame}
\frametitle{Elliptic curve endomorphisms}

% \textbf{EC Endomorphisms:} 
\emph{Homomorphisms $\psi:E(\Fclosure)  \to E(\Fclosure)$ that ``know'' $E$ is an EC}\nl
\begin{itemize}
 \item 
$\psi(P+Q) = \psi(P)+\psi(Q)$
\item $\psi(0) = 0$\pause
\item When $P=(x,y)$,
$\psi(x,y) = (r(x,y),s(x,y))$
for some rational functions $r,s$ over \F.\pause
\end{itemize}
EC endomorphisms are a \emph{ring} $\mathrm{END}_E$: $\psi_1\cdot \psi_2(P) \defeq \psi_1(\psi_2(P))$.
\end{frame}
\begin{frame}
\frametitle{Elliptic curve endomorphisms}

\hl{Example 1:}
$P-> c\cdot P$   for  integer c. \pause

% In fact, except for super-singular curves, all endomorphisms over $\F$ are combinations 
% of these two.
% \textit{(Let's assume $E$'s not super-singular from now on)}
\end{frame}

\begin{frame}
\frametitle{The Frobenius endomorphism}

\hl{Example 2:}
 $\phi(x,y)=(x^p,y^p)$.\pause
 \begin{itemize}
  \item Not the identity map because we're looking at points over \Fclosure.\pause
  \item It's really an endomorphism, basically cause $(A+B)^p = A^p+B^p$:\pause
  \[y^2-x^3-ax-b=0\]\pause
  \[\Rightarrow \left(y^2-x^3-ax-b\right)^p=0\]\pause
  \[\Rightarrow (y^p)^2-(x^p)^3-a(x^p)-b =0\]\pause
  \[\Rightarrow (x^p,y^p)\in E.\]

 \end{itemize}

\end{frame}



\begin{frame}
% \frametitle{First we find a root in $\mathrm{END}_E$..}
$\phi(x,y)=(x^p,y^p)$. $L(X)=X^2-aX+p$.\nlnp
% Reminder: $L

\hl{Lemma:} $\phi$ is a ``root'' of $L$.\nl
I.e., $\phi^2-a\cdot \phi + p$ is the all zero map.\nl
i.e.,  $\forall (x,y)\in E$:
\[ (x^{p^2},y^{p^2})-a\cdot (x^p,y^p)+p\cdot (x,y)=0_E\]\nl

maybe interesting for snark optimizers: 

$p\cdot (x,y) = a\cdot (x^p,y^p) - (x^{p^2},y^{p^2})$
\end{frame}

% \begin{frame}
% \textit{Interesting proof elements of lemma:}
% 
% $E_n\defeq \set{P\in E,n\cdot P=0}$. When $gcd(n,p)=1$, $E_n=\mathbb{Z}_n\times \mathbb{Z}_n$ .\nl
% $\rightarrow$ $\phi$ action on $E_n$ can be described by $2\times 2$ matrix $\phi_n$ over $\mathbb{Z}_n$.\nl
%  Can show: 
% $\mathrm{det}(\phi_n) = p\; \mathrm{mod}\; n$,   $\mathrm{trace}(\phi_n) = a\; \mathrm{mod} \;n$\nlnp
% Using Cayley-Hamilton follows that $L(\phi_n)=0$.
% 
% 
% \end{frame}
\begin{frame}
 $\phi$ is a ``root'' of $L$. Left to show it corresponds to a complex imaginary number that is also a root of $L$.\nl
 This is where \emph{complex multiplication} comes in.
\end{frame}

\begin{frame}
\frametitle{Complex multiplication over characteristic $p$}
\hl{Thm:}There is an isomorphism $T$ from
 $\mathrm{END}_E$ to a ring
$R_E = \mathbb{Z}+\mathbb{Z}\cdot d$, for some $d\in \mathbb{C}\setminus \mathbb{R}$.\nl
% (p. 314, a = c*sqrt(-d) for pos integer c)

\[L(T(\phi))=T(L(\phi))=T(0_{\mathrm{END}_E})=0\]\nl
So $\omega\defeq T(\phi)$ is a root of $L$!\nl
We showed before $\omega \notin \mathbb{Z}$ so must have 
$\omega\notin \mathbb{R}$ and we're done!
\end{frame}
\begin{frame}
\frametitle{Historical context}
Where did this last thm come from?
Torus/curve equivalence over complex numbers.\nl
$L$ is related to the zeta function of the elliptic curve, and RH for curve actually shows
$|\omega_1|=|\omega_2|=\sqrt p$. 

\end{frame}



\begin{frame}
 \hl{Sources:}
 \begin{itemize}
  \item 

 \emph{Elliptic curves number theory and cryptography} - Lawrence C. Washington.
  \item 
 \emph{The Riemann Hypothesis in Characteristic $p$ in Historical Perspective} - Peter Roquette
 \end{itemize}
 \vspace{0.5in}
  Thanks to Aztec crypto team for comments!
\end{frame}










% \begin{frame}
% Sidenote: exists curves over extension field where$ (x^p,y^p) = b(x,y)$ for some b
% i.e. frob is integer
% \end{frame}

\end{document}
