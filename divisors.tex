\documentclass[shadesubsections,compress,14pt,mathserif]{beamer}
\usepackage[danish]{babel}	
\usepackage{tikz}
\usetikzlibrary{shapes, positioning}
\usenavigationsymbolstemplate{}
\usepackage{pgfplots}
\usepackage[absolute,overlay]{textpos}

%\usepackage[T1]{fontenc}
%\usepackage{fourier}
% Dokumentets sprog
%\usepackage{mathtools}
%\usepackage{pxfonts}
\usepackage{eulervm}
% Class options include: notes, notesonly, handout, trans,
%                        hidesubsections, shadesubsections,
%                        inrow, blue, red, grey, brown

% Theme for beamer presentation.
%\usepackage{beamertheme} 
% Other themes include: beamerthemebars, beamerthemelined, 
%                       beamerthemetree, beamerthemetreebars  
 \newcommand{\prot}{\mathbf{P}}
\newcommand{\adv}{\ensuremath{\mathcal A}}
\newcommand{\aggdeg}[1]{\mathfrak{d}(#1)}
\renewcommand{\deg}{\mathrm{deg}}
\newcommand{\xor}{\ensuremath{\oplus}}
\newcommand{\plonk}{\ensuremath{\mathcal{P} \mathfrak{lon}\mathcal{K}}}
\newcommand{\F}{\ensuremath{\mathbb F}}
\newcommand{\set}[1]{\ensuremath{\left\{#1\right\}}}
\newcommand{\sett}[2]{\ensuremath{\left\{#1\right\}_{#2}}}
\newcommand{\enc}[1]{\ensuremath{\left[#1\right ]}}
\newcommand{\cm}{\ensuremath{\mathsf{cm}}}
\newcommand{\open}[1]{\ensuremath{\mathsf{open}(#1)}}
\newcommand{\verify}[1]{\ensuremath{\mathsf{verify}(#1)}}
\newcommand{\defeq}{\ensuremath{:=}}
\newcommand{\helper}{\ensuremath{\mathcal{H}}}
\newcommand{\ver}{\ensuremath{\mathcal{V}}}
\newcommand{\prv}{\ensuremath{\mathcal{P}}}
 \newcommand{\polysofdeg}[1]{\F_{< #1}[X]}
 \newcommand{\polys}{\F[X]}
\newcommand{\acc}{{\mathbf{acc}}}
\newcommand{\ideal}{\mathbf{I}}
\newcommand{\gen}{\alpha}
\newcommand{\plookup}{\mathsf{plookup}}
%\setbeamersize{text margin left=3mm,text margin right=3mm} 
\title{\large{Into the weeds of EC pairings}}    % Enter your title between curly braces
\author{\small{Ariel Gabizon}\\                 % Enter your name between curly braces
\tt{\footnotesize{\textbf{Aztec})                                        } }      }% Enter your institute name between curly braces
\date{}                    % Enter the date or \today between curly braces
%\usefonttheme{professionalfonts}
%\usefonttheme[onlymath]{serif}
\begin{document}
\boldmath
% Creates title page of slide show using above information
\begin{frame}
  \titlepage
\end{frame}



\begin{frame}
\frametitle{Divisors on $k(X)$}   % insert frame title between curly braces

% \textbf{Preprocessing/inputs:} Predefined polynomials $g_1,\ldots,g_t\in \polysofdeg{d}$\\
% \textbf{Range:} $H\subset\F$.\\ \pause
% \vspace{0.4in}
% \textbf{Protocol:}
$P\defeq  k\cup\infty$.\pause \\ 
\vspace{0.2in}
A divisor is a formal sum
$$D=\sum_{a\in P}d_a \cdot [a]$$
where $d_a\in\mathbb{Z}$ is non-zero except for finitely many $a$.

%  \begin{itemize}
% 
% \item  
% \item At end, $\ver$ asks $\ideal$ if some identity holds between $\set{f_1,\ldots,f_\ell,g_1,\ldots,g_t}$   \textbf{\textit{on $H$}}. 
% 
% \end{itemize}
% 
\end{frame}
\begin{frame}
\frametitle{Evaluating at $\infty$ via projective space}   

Evaluating $f(x)=\frac{x^2+x+1}{3x^2+1}$ at $\infty$:\\ \pause
\vspace{0.2in}
\textbf{Homogenize:} $--> \frac{x^2+xz+z^2}{3x^2+z^2}$ \\ \pause

\vspace{0.2in}
\textbf{Evaluate at $(x,z)=(1,0)$:}$\frac{1}{3}$
\end{frame}
\begin{frame}
\frametitle{Divisors of functions}   % insert frame title between curly braces

$f\in k(X)$
$$div(f)=\sum o_a(f)\cdot [a]$$
where $o_a(f)$ is the order of $f$ at $a$:\pause
$$f=(x-a)^{o_a(f)} (g(x))$$ 
where $g(a)\neq 0,\infty$ \\ \pause
\vspace{0.2in}
How to compute $o_{\infty}(f)$? If $f=g/h$ for polys $g,h$, $o_{\infty}(f) = deg(h)-deg(g)$


\end{frame}
\begin{frame}
\textbf{example:}
$$f=\frac{(X-1)^2(X-2)}{X-3}$$
$div(f)=2\cdot[1]+[2]-[3]-2[\infty]$.\pause

Define $deg(D)\defeq \sum_{a\in P} d_a$.\\  
For $f\in k(X)$ we always have $deg(div(f))=0$.

\end{frame}
\begin{frame}
 \frametitle{The divisor class group}
 \begin{itemize}
  \item The set of divisors is a group under coordinate wise addition\pause
  \item The set of divisors of degree zero is a subgroup $Div^0$ under this rule.\pause
  \item If $D=div(f)$ for $f\in k(x)$ we call $D$ a principal divisor.\pause
  \item The \emph{divisor class group of degree 0} is: $Div^0/$(principal divisors).
  
 \end{itemize}
Is this an interesting group?\pause
No, its trivial!
But this gets more interesting when we do it over an elliptic curve instead of a field.
\end{frame}
\begin{frame}
Suppose our curve $E$ is $y^2=x^3-x$.
Instead of $k(X)$ we'll work now over $H\defeq k(x,y)/(y^2-x^3-x)$. \\ \pause
\vspace{0.2in}
For example in $H$, $x=y^2\cdot\frac{1}{x^2-1}$.\\ \pause
\vspace{0.2in}
Now, a divisor is $D=\sum_{P\in E}d_j [P]$,
and for $f\in H$
$div(f)=\sum_{P\in E} o_P(f)[P]$\\ \pause
How to compute $o_P(f)$? \\
$$f=u^{o_P(f)} \cdot g$$ 
for $g$ with $g(P)\neq 0,\infty$
and $u$ with $o_P(u)=1$.

\end{frame}
\begin{frame}
It can be shown, like in $k(X)$ we always have $deg(div(f))=0$.\\ \pause
\vspace{0.2in}
 \textbf{Example:$f=x$}
 Compute $div(x)$.
 Can be shown $o_{\infty}(x) =-2$,$o_{(0,0)}(y)=1$.\\ \pause
 \vspace{0.2in}
 Since $x=y^2\cdot\frac{1}{x^2-1}$, we have $o_{(0,0)} (x)= 2$.\\
 \vspace{0.2in}
So $div(x) = 2([0,0]) - 2[\infty]$. 
 
\end{frame}
\begin{frame}
\frametitle{The cool theorem}
As before, we can define $C\defeq Div^0/(\textrm{principal divisors})$.\\ \pause
\vspace{0.2in}
It turns out $C$ is isomorphic to $E$ as a group!\\ \pause
\vspace{0.2in}
\end{frame}
\begin{frame}
\textbf{Proof sketch:}
We will show that every divisor $D$ of degree zero 
can be written as $D= div(g)+ [P]-[\infty]$.\\ \pause
\vspace{0.2in}
The idea is that divisors of line functions allow us to compress two points into one:
If we have $[P_1]+[P_2]$ as part of divisor
and $l(x,y)$ is the line passing through $P_1,P_2$
then 
$$div(l)=[P_1]+[P_2]+[P_3]-3[\infty]$$\\ \pause
So can switch: $[P_1]+[P_2]--> div(l)-[P_3]-3\cdot [\infty]$




\end{frame}

\end{document}
