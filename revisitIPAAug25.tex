\documentclass[shadesubsections,compress,14pt,mathserif]{beamer}
\usepackage[danish]{babel}	
\usepackage{tikz,circuitikz}
\usetikzlibrary{shapes, positioning}
\usenavigationsymbolstemplate{}
\usepackage{xcolor,pgfplots,bm}
\usepackage[absolute,overlay]{textpos}
\usepackage{amsthm,amsfonts}
%\usepackage[T1]{fontenc}
% \usepackage{fullpage}
% Dokumentets sprog
%\usepackage{mathtools}
%\usepackage{pxfonts}
\usepackage{eulervm}
\usepackage[export]{adjustbox}
\everymath{\color{purple}}
\definecolor{darkred}{rgb}{0.75, 0, 0.25} 
% Class options include: notes, notesonly, handout, trans,
%                        hidesubsections, shadesubsections,
%                        inrow, blue, red, grey, brown

% Theme for beamer presentation.
%\usepackage{beamertheme} 
% Other themes include: beamerthemebars, beamerthemelined, 
%                       beamerthemetree, beamerthemetreebars  
\newcommand{\minus}{\scalebox{0.5}[1.0]{\( - \)}}
\newcommand{\adv}{\ensuremath{\mathcal A}}
\newcommand{\F}{\ensuremath{{\mathbb F}}}
\newcommand{\G}{\ensuremath{{\mathbb G}}}
\renewcommand{\P}{\ensuremath{{\mathbb P}}}
\newcommand{\Z}{\ensuremath{{\mathbb Z}}\xspace}
\newcommand{\Fclosure}{\ensuremath{{\overline{\mathbb{F}}}_p}}
\newcommand{\set}[1]{\ensuremath{\left\{#1\right\}}}
\newcommand{\bin}[1]{\ensuremath{\set{0,1}^{#1}}}
\newcommand{\cube}{\ensuremath{\bin^n}}

\newcommand{\sett}[2]{\ensuremath{\left\{#1\right\}_{#2}}}
\newcommand{\enc}[1]{\ensuremath{\left[#1\right ]}}
% \newcommand{\kzg}[1]{\ensuremath{\enc{#1(x)}}}
\newcommand{\cm}{\ensuremath{\mathsf{cm}}}
\newcommand{\kzg}[1]{\cm(#1)}
\newcommand{\open}[1]{\ensuremath{\mathsf{open}(#1)}}
\newcommand{\verify}[1]{\ensuremath{\mathsf{verify}(#1)}}
\newcommand{\defeq}{\ensuremath{:=}}
\newcommand{\helper}{\ensuremath{\mathcal{H}}}
\newcommand{\ver}{\ensuremath{\mathcal{V}}}
\newcommand{\prv}{\ensuremath{\mathcal{P}}}
 \newcommand{\polysofdeg}[1]{\F_{< #1}[X]}
%  \newcommand{\endoss}{\ensuremath{\mathrm{END}_E}}
 \newcommand{\hl}[1]{\textbf{\textit{#1}}}
 \newcommand{\polys}{\F[X]}
\newcommand{\acc}{{\mathbf{acc}}}
\newcommand{\rej}{{\mathbf{rej}}}
\newcommand{\ideal}{\mathbf{I}}
\newcommand{\gen}{\alpha}
\newcommand{\spac}{\\  \vspace{0.2in} \noindent}
\newcommand{\polylog}{\ensuremath{\mathsf{polylog}}\xspace}
% \renewcommand{\bf}{\begin{frame}}
% \newcommand{\ef}{\end{frame}}
%\setbeamersize{text margin left=3mm,text margin right=3mm}  
\newcommand{\nl}{\\ \pause \vspace{0.2in}}
\newcommand{\nlnp}{\\ \vspace{0.2in}}
\newcommand{\stitle}[1]{{\large{\textcolor{purple}{\emph{#1}}}}}
\DeclareMathAlphabet{\mathpgoth}{OT1}{pgoth}{m}{n}	
\newcommand{\cq}{\mathpgoth{cq} }
\newcommand{\cqstar}{\ensuremath{\mathpgoth{cq^{\mathbf{*}} }}\xspace}
\newcommand{\flookup}{\ensuremath{\mathsf{\mathpgoth{Flookup}}}\xspace}
\newcommand{\baloo}{\ensuremath{\mathrm{ba}\mathit{loo}}\xspace}
% \newcommand{\caulkp}{\ensuremath{\mathsf{\mathrel{Caulk}\mathrel{\scriptstyle{+}}}}\xspace}
\newcommand{\caulk}{\ensuremath{\mathsf{Caulk}}\xspace}
\newcommand{\plookup}{\ensuremath{\mathpgoth{plookup}}\xspace}
\newcommand{\srs}{\ensuremath{\mathsf{srs}}}
\newcommand{\tablegroup}{\ensuremath{\mathbb{H}}\xspace}
\newcommand{\V}{\ensuremath{\mathbf{V} }\xspace}
\newcommand{\zfin}{\ensuremath{z_{\mathrm{final}}}}
\newcommand{\rel}{\ensuremath{\mathcal{R}}}
\newcommand{\vk}{\ensuremath{\mathrm{vk} }}
\newcommand{\repr}{\ensuremath{\mathrm{repr} }}
\newcommand{\numreads}{\ensuremath{\mathbf{numreads} }}
\newcommand{\add}{\ensuremath{\mathbf{add} }}
\newcommand{\adds}{\ensuremath{\mathbf{adds} }}
\newcommand{\cnt}{\ensuremath{\mathpgoth{t} }}
\newcommand{\addcount}{\ensuremath{\mathpgoth{at} }}
\renewcommand{\read}{\ensuremath{\mathbf{read} }}
\newcommand{\reads}{\ensuremath{\mathbf{reads} }}
\renewcommand{\note}{\ensuremath{\mathfrak{n} }}
\newcommand{\vknext}{\ensuremath{\mathrm{vk_{next}} }}
\newcommand{\vkcur}{\ensuremath{\mathrm{vk_{cur}} }}
\newcommand{\args}{\ensuremath{\mathrm{args} }}
\newcommand{\stack}{\ensuremath{\mathsf{stack} }}
\newcommand{\argscur}{\ensuremath{\mathrm{args_{cur}} }}
\newcommand{\argsnext}{\ensuremath{\mathrm{args_{next}} }}
% \newcommand{\caulk}{{\mathsf{Caulk}}}
% \newcommand{\caulkp}{{\mathsf{\mathrel{Caulk}\mathrel{\scriptstyle{+}}}}}
\newcommand{\bigspace}{\ensuremath{\mathbb{V}}}
\newcommand{\mle}[1]{\ensuremath{\hat{#1}}}
\newcommand{\eq}{\ensuremath{\mathbf{eq}}}
\newcommand{\sumi}[1]{\sum_{i< #1}}
\newcommand{\com}{\ensuremath{\mathsf{com}}}
\newcommand{\X}{\ensuremath{\mathbf{X}}}

%\setbeamersize{text margin left=3mm,text margin right=3mm} 
\title{\large{IPA as sumcheck}}    % Enter your title between curly braces
\author{\small{Ariel Gabizon (based on work with Liam Eagen)}\\                 % Enter your name between curly braces
\tt{\footnotesize{Aztec Labs}                                       } }      % Enter your institute name between curly braces
\date{}                    % Enter the date or \today between curly braces
%\usefonttheme{professionalfonts}
%\usefonttheme[onlymath]{serif}
\begin{document}
\boldmath
% Creates title page of slide show using above information
\begin{frame}
  \titlepage
\end{frame}


\begin{frame}
 \frametitle{Main Goal:}
 
 \begin{itemize}
  \item Reduce linear verifier time from IPA

 \end{itemize}
\end{frame}
\begin{frame}
 \frametitle{Polynomials over \G}
 
\end{frame}
\begin{frame}
 \frametitle{Multilinears and vectors}

$n=2^k$.\nlnp
$f=(f_0,\ldots,f_{n-1})\in \F^n,z\in \F^k$
\[\mle{f}(z)\defeq \sum_{i<n} \eq(i,z) f_i\]\pause
(can do same for $G\in \G^n$)
 \end{frame}
\begin{frame}
 \frametitle{MLPCS based on IPA:}
\textbf{Setup:}
Chooose random non-zero $G=(G_0,\ldots,G_{n-1})\in \G^n,P\in \G$.\nl
\textbf{Commitment:} $f\in \F^n$, $\com(f)=\sum_{i<n}f_i G_i$.\nl
\textbf{Openings:} next slide.
 \end{frame}
\begin{frame}
Given $\cm\in \G,z\in \F^k,v\in \F$ want to prove $\com(f)=\cm$ and $\mle{f}(z)=v$.\nl
Define the polynomial 
\[A(\X)\defeq    \mle{f}(\X) \mle{G}(\X) +  \eq(\X,z) \mle{f}(\X) P\]\pause
 When claim holds:
\[\sum_{b\in \bin{k}}\mle{f}(b) \mle{G}(b)+\eq(b,z)\mle{f}(b)P\]
\[= \sumi{n} f_i G_i + \eq(i,z) f_i P = \cm + \mle{f}(z) P\]\pause
\end{frame}
\begin{frame}
$\prv$ and $\ver$ will run sumcheck on $A$ with target value $\cm+vP$.\nl
At end of sumcheck, $\ver$ needs to evaluate $A(r)=\mle{f}(r)\mle{G}(r) + \eq(r,z)\mle{f}(r)P$ for some $r\in \F^k$.\nl
\textbf{In IPA:} $\ver$ computes $\eq(r,z),\mle{G}(r)$. $\prv$ \emph{simply sends} $a=\mle{f}(r)$.\nl
\textit{The strange (and useful) thing:} When $\G$ has hard discrete-log this is sound!\nl
\emph{Drawback:} Computing $\mle{G}(r)$ is $n$-size MSM for $\ver$!
 \end{frame}
 \begin{frame}
  \frametitle{Mitigation from Halo: defer MSM}
  Note
  \[\mle{G}(r) = \sumi{n} \eq(i,r) G_i,\]
  is the \emph{commitment to $\eq(\X,r)$!}\nl
  Can use this to reduce multiple evaluation claims $G(r_i)=V_i$ into one.
 \end{frame}

 \begin{frame}
  \frametitle{$\prv$ proving correctness of $\mle{G}(r)$}
\textit{Observation: If we have mlPCS for field-valued multilinears, where all ops on $f$'s vals are $\F$-linear, can also use on \emph{group valued} multilinear $G$. - e.g. Basefold}

 \end{frame}
 \begin{frame}
  \frametitle{Correlated agreement theorem:}

 \end{frame}
\end{document}
